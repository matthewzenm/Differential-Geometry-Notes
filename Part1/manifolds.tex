\chapter{Smooth Manifolds}

In this chapter we will introduce the basic object of differential geometry: smooth (or differentiable) manifolds and smooth maps.
In the meanwhile, we will introduce the main idea of differential geometry: linearization.
To be specific, we will introduce tangent spaces and tangent maps, i.e.\ the linearization of smooth maps.
The notion of manifolds with boundary and submanifolds will be discussed as well.

\section{Category of Smooth Manifolds}

Feel free if you know nothing about category theory, the section title just indicates we will define the notions of smooth manifolds and smooth maps.
Throughout the note, we will use ``smooth'' to mean being $C^\infty$.

We first have a topological object.
\begin{defn}
    Let $n\in\mathbb{N}$, a \emph{topological manifold} $M^n$ is a second countable Hausdorff space that is locally Euclidean, i.e.\ for any $p\in M$, there is an open neighborhood $U$ of $p$ and a map $\varphi_U:U\to\mathbb{R}^n$ that maps $U$ to an open set in $\mathbb{R}^n$ and homeomorphically onto its image.
    $(U,\varphi_U)$ is called a \emph{chart}.
    $n$ is called the \emph{dimension} of the manifold, denoted by $\dim{M}$.
    We will omit the superscript that indicates dimension if the dimension is clear.
\end{defn}

By gluing chart smoothly, we can have the notion of a smooth manifold.
\begin{defn}
    Let $M$ be a topological manifold, two charts $(U,\varphi_U)$ and $(V,\varphi_V)$ are called \emph{compatible} if both $\varphi_V\circ\varphi_U^{-1}$ and $\varphi_U\circ\varphi_V^{-1}$ are smooth.
    If compatible charts $\{(U_i,\varphi_i)\}$ consist an open cover of $M$, then it is called an \emph{atlas}.
    A maximal atlas is called a \emph{smooth structure}.
    A topological manifold together with a smooth structure is called a \emph{smooth manifold}.
\end{defn}

\begin{rem}
    \begin{enumerate}[(1)]
        \item Not all topological manifolds admit a smooth structure, see for isntance, \cite{Kervaire}.
        \item Dimension of manifolds are well-defined, that is, there is no diffeomorphisms between open subsets of $\mathbb{R}^m$ and $\mathbb{R}^n$ if $m\neq n$.
        The proof requires topological invariants, such as homology groups.
        See for instance, \cite[Theorem~2.55]{LeeTM}.
        \item We will assume all the manifolds occur in our note are path connected unless we specially claim.
        \item Notice that we only need to give an atlas of a manifold to define a smooth structure.
    \end{enumerate}
\end{rem}

We now present some examples of smooth manifolds.

\begin{eg}
    \begin{enumerate}[(1)]
        \item $\mathbb{R}^n$ itself is a smooth manifold, with single chart $\operatorname{id}$.
        \item Any open subset $U$ of a smooth manifold $M$.
        If $\{U_i,\varphi_i\}$ is an atlas of $M$, then $\{U_i\cap U,\varphi_i|_{U_i\cap U}\}$ is an atlas of $U$.
        In particular, any open set of $\mathbb{R}^n$ is a smooth manifold.
    \end{enumerate}
\end{eg}

\begin{eg}
    Sphere $\mathbb{S}^n=\{x\in\mathbb{R}^{n+1}:\ |x|=1\}$.
    An atlas is given by stereographic projections from north pole and south pole.
    We give the formula explicitly.
    Let $U=\mathbb{S}^n-\{(0,\cdots,0,1)\}$, $V=\mathbb{S}^n-\{(0,\cdots,0,-1)\}$, then
    \begin{gather*}
        U\xrightarrow{\varphi}\mathbb{R}^n,\ (x_1,\cdots,x_{n+1})\mapsto\left(\frac{x_1}{1-x_{n+1}},\cdots,\frac{x_n}{1-x_{n+1}}\right),\\
        V\xrightarrow{\psi}\mathbb{R}^n,\ (x_1,\cdots,x_{n+1})\mapsto\left(\frac{x_1}{1+x_{n+1}},\cdots,\frac{x_n}{1+x_{n+1}}\right).
    \end{gather*}
    It's easy to see $(U,\varphi)$ and $(V,\psi)$ are compatible.
\end{eg}

\begin{eg}
    \begin{enumerate}[(1)]
        \item If $M$, $N$ are smooth manifolds, then $M\times N$ is a smooth manifold with atlas $\{U_i\times V_j,\varphi_i\times\psi_j\}$, provided $\{U_i,\varphi_i\}$ and $\{V_j,\psi_j\}$ are atlases of $M$ and $N$ respectively.
        \item The torus $\mathbb{T}^n$ is defined by $\mathbb{S}^1\times\cdots\times\mathbb{S}^1$ for $n$ times.
    \end{enumerate}
\end{eg}

To complete the definition of category of smooth manifolds, we need to define smooth maps.
\begin{defn}
    A map $f:M\to N$ is called \emph{smooth} at $p\in M$, if there exists chart $(U,\varphi)$ of $p$ and $(V,\psi)$ of $f(p)$, such that $\psi\circ f\circ\varphi$ is smooth on $\varphi(U)$.
    If $f$ is smooth everywhere, we simply say $f$ is smooth.
    If $N=\mathbb{R}$, then we call $f$ a smooth \emph{function}, and we denote the set (actually $\mathbb{R}$-algebra) of smooth functions on $M$ by $C^\infty(M)$.
\end{defn}

The isomorphisms in the category of smooth manifolds are called diffeomorphisms, and we repeat the definition here.
\begin{defn}
    A smooth map $f:M\to N$ is called a \emph{diffeomorphism} if there exists a smooth $g:N\to M$, such that $g\circ f=\operatorname{id}_M$ and $f\circ g=\operatorname{id}_N$.
\end{defn}

At the end of the section, we discuss partition of unity.

\begin{defn}
    A collection of subsets $\{X_i\}_{i\in I}$ of topological space $X$ is called \emph{locally finite} if for any $x\in X$, there is a neighborhood $U$ of $x$ such that only finite many $i\in I$ satisfy $U\cap X_i\neq\varnothing$.
\end{defn}

\begin{defn}
    The \emph{support} of a smooth function $f\in C^\infty(M)$ is the closure of $\{p\in M:\ f(p)\neq 0\}$, denoted by $\supp{f}$.
\end{defn}

Now we can state the partition of unity theorem here, but we will not present the long and complicated proof.
For a proof, one can check~\cite[Theorem~2.23]{LeeSM}.

\begin{thm}[Partition of unity]
    Let $M$ be a smooth manifold with an open cover $\{U_i\}_{i\in I}$.
    Then there exists a family of smooth functions $\{\varphi_i\}_{i\in I}$ satisfying
    \begin{enumerate}[(1)]
        \item $0\leq\varphi_i\leq 1$;
        \item $\supp\varphi_i\subset U_i$ for each $i\in I$;
        \item $\{\supp\varphi_i\}_{i\in I}$ is locally finite;
        \item $\sum_{i\in I}\varphi_i(p)=1$ for $p\in M$.
    \end{enumerate}
\end{thm}

\begin{rem}
    By (3), the summation in (4) is a finite sum, so we will not encounter convergence problem here.
\end{rem}

Partition of unity plays an important role in defining many geometric objects, such as integral, metric, connection and so on.
But any of them is beyond our scope in this section, so we have to postpone the illustration of applications of partition of unity.

\section{Tangent Spaces and Tangent Maps}

In this section we discuss the linearization of smooth maps, the main idea of differential geometry.

We first define tangent spaces.
In modern textbooks, tangent vectors are defined to be derivatives on the germ of smooth functions.
This definition can show directly that tangent vectors consists a vector space.
However, we will adopt traditional definition via curves on manifolds, which will make our calculation more convenient.

\begin{defn}
    The \emph{germ of smooth functions} at $p\in M$ is a ring defined by
    \[\{(f,U)|\ U\subset M\text{ open, }f\in C^\infty(U)\}/\sim,\]
    where $(f,U)\sim(g,V)$ if there exists a $W\subset U\cap V$ such that $f|_W=g|_W$.
    Ring operations are defined in natural way.
    Elements in $C^\infty_p(M)$ are denoted by $f_p$ if it has representative element $f$.
\end{defn}

\begin{defn}
    Let $\gamma:[0,1]\to M$ be a smooth curve, i.e.\ a smooth map, with $\gamma(0)=p$.
    Define the \emph{tangent vector at $p$ along $\gamma$} to be a map
    \begin{align*}
        \dot\gamma(0):C^\infty_p(M)&\to\mathbb{R}\\
        f_p&\mapsto(f\circ\gamma)'(0).
    \end{align*}
    Then we define the \emph{tangent space at $p$} to be
    \[T_pM:=\{\dot\gamma(0)|\ \gamma:[0,1]\to M\text{ smooth, }\gamma(0)=p\}.\]
\end{defn}

\begin{rem}
    The assignment of curve to its tangent vector is not a bijection.
    For instance, on $T_0\mathbb{R}^2$, $y=0$ and $y-x^2=0$ yield same tangent vector at $(0,0)$.
\end{rem}

It's easy to see that
\begin{prop}
    We have \emph{Leibniz rule} 
    \[\dot\gamma(0)(f_pg_p)=f(p)\dot\gamma(0)(g_p)+g(p)\dot\gamma(0)(f_p).\]
    Therefore a tangent vector is a \emph{derivative} on $C^\infty_p(M)$.
\end{prop}

We will show that $T_pM$ has a real vector space structure.
In fact, we have an enhanced conclusion.
\begin{prop}\label{local tangent vector}
    Let $M^n$ be a smooth manifold, $(U,\varphi)$ be a chart containing $p$, $\varphi=(x^1,\cdots,x^n)$.
    Define $\left.\frac{\partial}{\partial{x^i}}\right|_p$ to be the tangent vector of $\sigma_i=\varphi^{-1}(\varphi(p)+te_i)$, where $e_i=(0,\cdots,1,\cdots,0)$ with only $i$-th component being $1$.
    Then we have
    \[T_pM=\Span\left\{\left.\frac{\partial{}}{\partial{x^i}}\right|_p\right\}\]
    as sets, so we can equip $T_pM$ a real vector space structure.
\end{prop}
\begin{proof}
    Notice that
    \[\left.\frac{\partial{}}{\partial{x^i}}\right|_pf=\frac{\partial}{\partial{x^i}}(f\circ\varphi^{-1})(p),\]
    the partial derivative on right hand side is the usual Euclidean partial derivative.
    Hence by chain rule, we have
    \begin{align*}
        \dot\gamma(0)f&=\left.\frac{\d{}}{\d{t}}\right|_{t=0}(f\circ\gamma)(t)\\
        &=\left.\frac{\d{}}{\d{t}}\right|_{t=0}(f\circ\varphi^{-1})\circ(\varphi\circ\gamma)(t)\\
        &=\left.\frac{\partial{}}{\partial{x^i}}\right|_{\varphi(p)}(f\circ\varphi^{-1})\left.\frac{\d{}}{\d{t}}\right|_{t=0}x^i(\gamma(t))\\
        &=\left.\frac{\d{}}{\d{t}}\right|_{t=0}x^i(\gamma(t))\left.\frac{\partial{}}{\partial{x^i}}\right|_pf.
    \end{align*}
    Therefore we have
    \[\dot\gamma(0)=\dot\gamma(0)(x^i)\left.\frac{\partial{}}{\partial{x^i}}\right|_p\in\Span\left\{\left.\frac{\partial{}}{\partial{x^i}}\right|_p\right\},\]
    and we obtain
    \[T_pM\subset\Span\left\{\left.\frac{\partial{}}{\partial{x^i}}\right|_p\right\}.\]
    The reverse inclusion is trivial.
\end{proof}

\begin{rem}
    The vector space structure of tangent space does not depend on the choice of chart, since for any charts $(U_1,\varphi_1)$ and $(U_2,\varphi_2)$, the Jacobian of transition function $\varphi_1\circ\varphi_2^{-1}$ is a vector space isomorphism.
    However, this is again rely on the well-definedness of dimension of manifolds, but we have assumed it automatically.
\end{rem}

Next we define tangent maps.
\begin{defn}
    Let $f:M\to N$ be a smooth map, $p\in M$.
    We define $f_{*p}:T_pM\to T_{f(p)}N$ by
    \[f_{*p}(v)(g)=v(g\circ f)\]
    for any $v\in T_pM$ and $g\in C^\infty_{f(p)}N$.
\end{defn}

\begin{rem}
    We often write $f_{*p}$ as $\d{f}|_p$ if $f$ is a smooth function.
    This coincides the notation for $1$-form we will define later.
\end{rem}

\begin{lem}
    On charts $(U,\varphi)$ of $p\in M$ and $(V,\psi)$ of $q\in N$, the smooth map $f:M\to N$ that $f(p)=q$ satisfies
    \[f_{*p}\left(\left.\frac{\partial{}}{\partial{x^i}}\right|_p\right)=\frac{\partial{}}{\partial{x^i}}(\psi\circ f\circ\varphi)^j(\varphi(p))\left.\frac{\partial{}}{\partial{y^j}}\right|_q.\]
\end{lem}
The proof is similar to Proposition~\ref{local tangent vector}.

Moreover, we have the chain rule.
\begin{prop}
    Let $f:M\to N$, $g:N\to P$ be two smooth maps, then for $p\in M$ we have
    \[(g\circ f)_{*p}=g_{*f(p)}\circ f_{*p}.\]
\end{prop}
\begin{proof}
    Let $v\in T_pM$, then for any $\varphi\in C^\infty_{g(f(p))}(M)$ we have
    \begin{align*}
        (g\circ f)_{*p}(v)(\varphi)&=v(\varphi\circ(g\circ f))\\
        &=v((\varphi\circ g)\circ f)\\
        &=(f_{*p}(v))(\varphi\circ g)\\
        &=(g_{*f(p)}(f_{*p}(v)))(\varphi)\\
        &=(g_{*f(p)}\circ f_{*p})(v)(\varphi).\qedhere
    \end{align*} 
\end{proof}

Since we linearize smooth maps, we can talk about rank of a map at a point.
\begin{defn}
    Let $f:M\to N$ be a smooth map and $p\in M$, the \emph{rank} of $f$ at $p$, denoted by $\rank_p{f}$, is defined by the rank of $f_{*p}$.
\end{defn}

\begin{defn}
    Let $f:M^m\to N^n$ be a smooth map.
    \begin{enumerate}[(1)]
        \item Assume $m\leq n$, if $f_{*p}$ is injective for any $p\in M$, then we call $f$ an \emph{immersion}.
        \item Assume $f$ is an immersion, if $f$ maps $M$ homeomorphically onto its image, then we call $f$ an \emph{embedding}.
        \item Assume $m\geq n$, if $f_{*p}$ is surjective for any $p\in M$, then we call $f$ a \emph{submersion}.
    \end{enumerate}
\end{defn}

\begin{rem}
    An equivalent description for immersion and submersion is that they are of constant maximal rank.
    Also notice that a map is both an immersion and submersion if and only if it is a diffeomorphism.
\end{rem}

\begin{eg}
    The definition of embedding $f:M\to N$ asks $f(M)$ is equipped with subspace topology from $N$, so $f$ being injective is not enough.
    For example, consider the injective map
    \[r:(-\pi,\pi)\to\mathbb{R}^2,\ t\mapsto(2\sin{t},\sin(2t)).\]
    The image of $r$ is the zero locus of $4y^2=x^2(4-x^2)$, it's easy to see it is compact.
    But $(-\pi,\pi)$ is not compact, hence $r$ does not map the interval homeomorphically onto its image.
\end{eg}

\section{Submanifolds}

In this section, we discuss submanifolds of an ambient manifold.
We first present the definition.
Then, we will discuss rank theorem, a powerful tool to discuss submanifolds.
We will use rank theorem to derive inverse function theorem and implicit function theorem, and discuss submanifolds and embedded submanifolds.

\begin{defn}
    Let $M$ be a smooth manifold, $\Sigma\subset M$ be a subset.
    $\Sigma$ is called a \emph{($k$-dimensional) submanifold} of $M$ if for any $p\in\Sigma$, there exists a chart $(U,\varphi)$ of $p$ in $M$ such that 
    \[\varphi(U\cap\Sigma)=\varphi(U)\cap\{(x^1,\cdots,x^k,0,\cdots,0)\in\mathbb{R}^n|\ x^1,\cdots,x^k\in\mathbb{R}\}.\]
    We define the \emph{codimension} of $\Sigma$ to be $n-k$, denoted by $\codim\Sigma=n-k$.
\end{defn}

We will not linger to give any examples.
On the contrary, we will discuss rank theorem at once.

\begin{thm}[Rank theorem]
    Let $f:M^m\to N^n$ be a smooth map that has constant rank $r(\leq m)$.
    For any $p\in M$, there exists charts $(U,\varphi)$ of $p$ and $(V,\psi)$ of $f(p)$, such that
    \[(\psi\circ f\circ\varphi^{-1})(x^1,\cdots,x^r,\cdots,x^m)=(x^1,\cdots,x^r,0,\cdots,0)\]
    on $\varphi(U)$.
\end{thm}

The proof of rank theorem is also complicated, and we just refer to~\cite[Theorem~4.12]{LeeSM}.
Rank theorem has important applications.
First, as corollary, we have inverse function theorem and implicit function theorem.
\begin{thm}[Inverse function theorem]
    Let $f:M\to N$ be a smooth map, $p\in M$.
    Suppose $f_{*p}$ is an isomorphism, then there exists a neighborhood $U$ of $p$ such that $f|_U$ is an diffeomorphism on $U$.
\end{thm}
\begin{proof}
    Since determinant function is smooth, $\det{f_{*p}}\neq 0$ in a neighborhood of $p$.
    Then by rank theorem, there exists sufficiently small charts $(U,\varphi)$ of $p$ and $(V,\psi)$ of $f(p)$ such that $\psi\circ f\circ\varphi^{-1}$ is identity on $\varphi(U)$.
    Thus $f|_U=\psi^{-1}\circ\varphi$ is a homeomorphism, and it is a diffeomorphism since $f$ is smooth.
\end{proof}

\begin{thm}[Implicit function theorem]
    Let $f:M^{k+n}\to\mathbb{R}^k$ be a smooth map.
    Suppose $f_{*p}$ is a submersion for any $p\in f^{-1}(0)$, then $f^{-1}(0)$ is an $n$-dimensional submanifold of $M$ (may not be connected).
\end{thm}
\begin{proof}
    For $p\in f^{-1}(0)$, since determinant function is smooth, there exists a neighborhood of $p$ such that the tangent map has constant rank.
    Hence by rank theorem, there exists sufficiently small charts $(U,\varphi)$ of $p$ and $(V,\psi)$ of $0$ such that
    \[(\psi\circ f\circ\varphi^{-1})(x^1,\cdots,x^k,x^{k+1},\cdots,x^{k+n})=(x^1,\cdots,x^k).\]
    Then $f^{-1}(0)$ restricts on $U$ has coordinate
    \[\varphi(f^{-1}(0)\cap U)=\{(0,\cdots,0,x^{k+1},\cdots,x^{k+n})\in\mathbb{R}^{n+k}\}\cap\varphi(U).\]
    Thus $f^{-1}(0)$ is an $n$-dimensional submanifold.
\end{proof}
\begin{rem}
    \begin{enumerate}[(1)]
        \item The value $0$ plays an unimportant role in the proof.
        In fact, we can replace $0$ by $r_0$ and consider $g(x)=f(x)-r_0$.
        \item The condition of $f_{*p}$ being a submersion for any $p\in f^{-1}(r_0)$ is seemingly hard to verify.
        However, we will prove later that such property holds \emph{almost everywhere} in $\mathbb{R}$.
        (In fact, $\mathbb{R}^k$ for all $k\in\mathbb{N}$. This is Sard's Theorem, but we only need $k=1$ case in this note.)
    \end{enumerate}
\end{rem}

We now characterize submanifolds.
\begin{defn}
    Let $f:\Sigma\to M$ be an immersion, then the image of $f$ is called an \emph{immersed submanifold}.
    Moreover, if $f$ is an embedding, the image of $f$ is called an \emph{embedded manifold}.
\end{defn}

\begin{prop}
    Any immersed manifold is locally an embedded manifold.
\end{prop}
\begin{proof}
    Only need to show any immersion is locally an embedding, and this is clear from rank theorem.
\end{proof}

\begin{prop}
    A subset of a smooth manifold is a submanifold if and only if it is an embedded submanifold.
\end{prop}
\begin{proof}
    Let $M$ be a smooth manifold, $S\subset M$ be a submanifold, then the inclusion $i:S\to M$ is an embedding.
    Conversely, let $f:\Sigma^k\to M^n$ be an embedding.
    Let $p\in f(\Sigma)$, $f(q)=p$, then by rank theorem, there exists charts $(U,\varphi)$ of $q$ and $(V,\psi)$ of $p$ such that
    \[(\psi\circ f\circ\varphi^{-1})(x^1,\cdots,x^k)=(x^1,\cdots,x^k,0,\cdots,0).\]
    Since $\Sigma\cong f(\Sigma)$ with latter equipped with subspace topology, $f(U)=\tilde{V}\cap f(\Sigma)$ for some open $\tilde{V}\subset M$.
    By shrinking $\tilde{V}$ (and $U$), we can assume $\tilde{V}\subset V$, then
    \[\psi(\tilde{V}\cap f(\Sigma))=\{(x^1,\cdots,x^k,0,\cdots,0)\in\mathbb{R}^n\}\cap\psi(\tilde{V}),\]
    hence $f(\Sigma)$ is a submanifold of $M$.
\end{proof}

Now we provide some examples of submanifolds.
\begin{eg}
    \begin{enumerate}[(1)]
        \item Open subsets of a smooth manifold.
        This is obvious.
        \item \emph{Surfaces} in $\mathbb{R}^3$.
        This means a submanifold of codimension $1$.
        In classical differential geometry, a surface means a $2$-dimensional immersed submanifold.
        That is, a smooth map $r:\Omega\subset\mathbb{R}^2\to\mathbb{R}^3$ with $r_{;1}\times r_{;2}\neq 0$.
        Generally, a submanifold of codimension $1$ (no matter what ambient manifold) is called a \emph{hypersurface}.
        \item \emph{Zero locus} of $f:M\to\mathbb{R}$ such that $f_{*p}$ has maximal rank for all $p\in f^{-1}(0)$.
        In particular, $n$-sphere $\mathbb{S}^n$ is the zero locus of $|x|^2-1=\sum_{i=1}^n(x^i)^2-1$.
        \item \emph{(Affine) algebraic manifolds}, i.e.\ \emph{smooth algebraic varieties}\footnote{Usually we only discuss algebraic varieties on algebraic closed fields, such as $\mathbb{C}$. But define varieties on arbitrary field does not violate the definition in algebraic geometry.}.
        Algebraic varieties are common zero locus of a family of polynomials $\{f^i\}_{i\in I}$.
        By Hilbert's basis theorem, we can assume the index set $I$ is finite (see for example, \cite[Theorem~B-1.16]{Rotman}).
        Hence we can have a differential smoothness condition that $f=(f^1,\cdots,f^{|I|})$ has maximal rank everywhere, instead of using commutative algebra.
    \end{enumerate}
\end{eg}

\section{Manifolds with Boundary}

\begin{defn}
    A \emph{(smooth) manifold with boundary} is (what?)
\end{defn}