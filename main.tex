% !TeX program = xelatex
% !TeX encoding = UTF-8

\documentclass[]{book}

\usepackage[paper=b5paper]{geometry}
\usepackage{amsmath}
\usepackage{amsthm}
\usepackage{amssymb}
\usepackage{mathrsfs}
\usepackage[shortlabels]{enumitem}
\usepackage{mathpazo}
% \usepackage{mathptmx}
\usepackage{fixdif}
\usepackage[colorlinks]{hyperref}

\theoremstyle{definition}
\newtheorem*{defn}{Definition}
\newtheorem{eg}{Example}[chapter]
\theoremstyle{plain}
\newtheorem{thm}[eg]{Theorem}
\newtheorem{prop}[eg]{Proposition}
\newtheorem{lem}[eg]{Lemma}
\newtheorem{cor}[eg]{Corollary}
\theoremstyle{remark}
\newtheorem{rem}[eg]{Remark}
\newtheorem*{pro}{Problem}

\DeclareMathOperator{\tr}{tr}
\DeclareMathOperator{\Span}{Span}
\DeclareMathOperator{\Sect}{Sect}
\DeclareMathOperator{\Ric}{Ric}
\DeclareMathOperator{\Scal}{Scal}
\newcommand{\R}[4]{R_{#2#3#4}{}^{#1}} % contravariant index is put the first
\newcommand{\Ricij}[2]{\mathrm{Ric}_{#2}{}^{#1}} % as above

\title{A Concise Note on Differential Geometry}
\author{Zeng Mengchen}
\date{Last Compile: \today}

\begin{document}
\maketitle

\frontmatter

\tableofcontents
\chapter{Preface}

\mainmatter

\part{Smooth Manifolds}

\part{Riemannian Geometry}
\chapter{Metric Structure of Riemannian Manifolds}

In this chapter we introduce the metric structure on a Riemannian manifold, i.e.\ using the metric tensor to make the manifold into a metric space.
We will prove that the topology induced by metric coincides with the topology carried by the manifold.
Then we will study the length-minimizing problem: which curve minimize the distance between two points?
The answer is two-sided: we will prove that length-minimizing curves are geodesics, and geodesics are \emph{locally} length-minimizing.

\section{Metric Structure on Riemannian Manifolds}

Let $(M,g)$ be a Riemannian manifold, $\gamma:[0,1]\to M$ be a regular curve (i.e.\ $\gamma$ is an immersion).
We define the length functional of regular curves.
\begin{defn}
    The \emph{length functional} $L[\cdot]$ on the set of regular curves is defined by
    \[L[\gamma]=\int_0^1\sqrt{g(\dot\gamma(t),\dot\gamma(t))}\d{t}.\]
\end{defn}

It is a well-known result from calculus that any regular curve can be repara\-metrized by arc-length.
From now on the word ``curve'' means a regular curve.

Now we define the distance between two points.
\begin{defn}
    Let $p,q\in M$ be two points on $M$, then we define their \emph{distance} by
    \[d(p,q)=\inf_{\gamma\in C_{p,q}}L[\gamma],\]
    where $C_{p,q}$ is the set of all regular curves joining $p$ and $q$.
\end{defn}

\begin{prop}
    The distance function $d:M\times M\to\mathbb{R}$ has the following properties:
    \begin{enumerate}[(1)]
        \item $d(p,q)\geq 0$, and $d(p,q)=0\iff p=q$;
        \item $d(p,q)=d(q,p)$;
        \item $d(p,r)\leq d(p,q)+d(q,r)$.
    \end{enumerate}
    Thus the distance function makes $M$ into a metric space.
\end{prop}
\begin{proof}
    Only need to show $d(p,q)=0\iff p=q$, all else are trivial.
    We assume $p\neq q$, need to show $d(p,q)>0$.
    Let $\gamma:[0,1]\to M$ be any curve joining $p$ and $q$.
    Choose a local chart $(U,\varphi)$ such that $\varphi(U)=B_r(0)$, $q\neq U$.
    By Jordan--Brouwer Separation Theorem, $\gamma$ must intersect $\partial U$ at $s:=\gamma(c)$.
    Then we have
    \[L[\gamma]\geq L[\gamma|_{[0,c]}]=\int_0^c\sqrt{g_{ij}\dot{x}^i(\gamma(t))\dot{x}^j(\gamma(t))}\d{t}.\]
    Regarding $g:\overline{U}\times\mathbb{S}^{n-1}\to\mathbb{R}$, $g$ is a continuous function on a compact set, thus it attains its minimum $g(x)(v,v)\geq m$, and $m>0$ since $v\in\mathbb{S}^{n-1}\neq 0$.
    Thus we have
    \begin{equation}
        L[\gamma]\geq L[\gamma|_{[0,c]}]\geq m\int_0^c|\dot{x}(\gamma(t))|\d{t}\geq mr>0.\label{estm of dist}
    \end{equation}
    $mr$ does not depend on $\gamma$, hence $d(p,q)\geq mr>0$.
\end{proof}

However, the metric space topology is nothing but the original topology carried by the manifold.

\begin{prop}\label{metric and manifold topology}
    The metric space topology on $M$ coincides with the manifold topology.
\end{prop}

We first need a lemma.
\begin{lem}
    The distance function to $p$ defined by $r(q)=d(p,q)$ is continuous with respective to the manifold topology.
\end{lem}
\begin{proof}
    Since manifolds satisfy the second countable axiom, the Sequence Lemma holds.
    Then it's equivalent to show for any $q_i\to q$ in manifold topology, we have $r(q_i)\to r(q)$.
    Without loss of generality we can assume $\{q_i\}\subset U$ and $(U,\varphi)$ is a local chart such that $\varphi(q)=0$, $\varphi(U)=B_r(0)$.
    Let $\delta$ be the Euclidean metric on $B_r(0)$, then by regarding $g$ as a continuous function on $\overline{U}\times\mathbb{S}^{n-1}$ again, we have $g\leq M\delta$ for some $M>0$.
    By assumption, $q_i\to q$ in manifold topology implies $L_\delta[\psi_i]\to 0$, where $\psi_i(t)=t\varphi(q_i)$, the radial line joining $\varphi(q)$ and $\varphi(q_i)$ in $\varphi(U)=B_r(0)$.
    Let $\varphi^{-1}\psi_i=\gamma_i$, then we have
    \[L_g[\gamma_i]=\int_0^1\sqrt{g(\dot\gamma_i(t),\dot\gamma_i(t))}\d{t}\leq M\int_0^1\sqrt{\delta(\dot\psi_i(t),\dot\psi_i(t))}\d{t}\leq ML_\delta[\psi_i].\]
    Since $r$ is Lipschitz, i.e.\ $|r(q)-r(s)|\leq d(q,s)$, we have
    \[d(q_i,q)\leq L_g[\gamma_i]\leq ML_\delta[\psi_i]\to 0,\]
    hence $r$ is continuous.
\end{proof}

\begin{proof}[Proof of Proposition~\ref{metric and manifold topology}]
    Since distance function is continuous, metric balls are open in manifold topology.
    Now we prove the converse.

    Let $U$ be open with respective to manifold topology.
    Let $p\in U$, $V$ be a neighborhood of $p$ so small that $\varphi(V)=B_r(0)$ for some $r>0$.
    The estimate~\eqref{estm of dist}~shows if $q\notin V$ then $d(p,q)\geq mr$ for some fixed $m>0$, then by taking contrapositive statement, we have $q\in V$ if $d(p,q)<mr$.
    Therefore $B_p(mr)\subset U$, then $U$ is open with respective to metric space topology.
\end{proof}

\section{Length-Minimizing Curves}

This section and the next is guided by the following problem:
\begin{pro}
    Let $p,q$ be two distinct points on a Riemannian manifold $(M,g)$, then what is the curve $\gamma$ satisfying $L[\gamma]=d(p,q)$?
\end{pro}

This section we will show if the length-minimizing curve exists, then it is a \emph{geodesic}.
Next section we will show if $p,q$ are sufficiently closed, \emph{the} geodesic joining $p$ and $q$ is length-minimizing.
Finally, the existence of length-minimizing curves is related to Hopf-Rinow Theorem, which we will discuss at the last section of the chapter.

To find the length-minimizing curve, we ``gather'' curves with same initial and end points, which is called a variation.
\begin{defn}
    Let $\gamma_0:[0,a]\to M$ be a curve, a \emph{variation} of $\gamma_0$ is a smooth map $\gamma:[0,a]\times(-\varepsilon,\varepsilon)\to M$ such that $\gamma(t,0)=\gamma_0(t)$.
    If $\gamma(0,s)=\gamma_0(0)$ and $\gamma(a,s)=\gamma_0(a)$ for any $s\in(-\varepsilon,\varepsilon)$, then we call the variation a \emph{proper variation}.
    We call $\left.\frac{\partial{}}{\partial{s}}\right|_{s=0}\gamma(s,t)=:V(t)$ the \emph{variation vector field}.
\end{defn}

Now we introduce the energy functional, which is easier to calculate.
\begin{defn}
    The \emph{energy functional} on the set of curves is defined by
    \[E[\gamma]=\int_0^a\frac{1}{2}|\dot{\gamma}(t)|^2\d{t},\]
    where $\gamma:[0,a]\to M$ is a regular curve.
\end{defn}

We will prove that a curve is energy-minimizing if and only if it is length-minimizing.
\begin{lem}
    For a curve $\gamma:[0,a]\to M$, we have
    \[L^2[\gamma]\leq 2aE[\gamma],\]
    with equality holds if and only if $|\dot\gamma(t)|=\mathrm{const}$.
\end{lem}
\begin{proof}
    This is Cauchy--Schwarz inequality.
\end{proof}

\begin{prop}\label{length-min to energy-min}
    If $\gamma$ is length-minimizing, then it is energy-minimizing.
\end{prop}
\begin{proof}
    Let $\tilde\gamma$ be another curve, then we have
    \[2aE[\gamma]=L^2[\gamma]\leq L^2[\tilde\gamma]\leq 2aE[\tilde\gamma].\qedhere\]
\end{proof}

Our aim is to prove the converse.
\begin{prop}\label{energy-min to length-min}
    If $\gamma$ is an energy-minimizing curve, then it is length-minimizing.
\end{prop}

To prove this, we need to differentiate the variation.
\begin{prop}[First variation formula]
    Let $\gamma(t,s)$ be a variation, define its energy $\displaystyle E(s)=\int_0^a\frac{1}{2}\left|\frac{\partial}{\partial{t}}\gamma(t,s)\right|^2\d{t}$, then we have
    \[E'(0)=\boxed{\langle V,\dot\gamma\rangle|^a_0}-\int_0^a\langle V(t),\nabla_{\dot\gamma_0(t)}\dot\gamma_0(t)\rangle.\]
    The boxed term is called \emph{boundary term}, and it vanishes when the variation is proper.
\end{prop}
\begin{proof}
    This is a calculation.
    We have
    \begin{align*}
        \frac{\d{}}{\d{s}}E(s)&=\int_0^a\left\langle\frac{\partial\gamma}{\partial{t}},\nabla_{\frac{\partial}{\partial{s}}}\frac{\partial\gamma}{\partial{t}}\right\rangle\d{t}\\
        &=\int_0^a\left\langle\frac{\partial\gamma}{\partial{t}},\nabla_{\frac{\partial}{\partial{t}}}\frac{\partial\gamma}{\partial{s}}\right\rangle\d{t}\\
        &=\int_0^a\frac{\partial}{\partial{t}}\left\langle\frac{\partial\gamma}{\partial{t}},\frac{\partial\gamma}{\partial{s}}\right\rangle-\left\langle\frac{\partial\gamma}{\partial{s}},\nabla_{\frac{\partial{}}{\partial{t}}}\frac{\partial\gamma}{\partial{t}}\right\rangle\d{t}.
    \end{align*}
    Take $s=0$, we obtain
    \begin{align*}
        E'(0)&=\int_0^a\frac{\partial{}}{\partial{t}}\langle V(t),\dot\gamma_0(t)\rangle-\langle V(t),\nabla_{\dot\gamma_0(t)}\dot\gamma_0(t)\rangle\d{t}\\
        &=\langle V,\dot\gamma_0\rangle|^a_0-\int_0^a\langle V(t),\nabla_{\dot\gamma_0(t)}\dot\gamma_0(t)\rangle\d{t}.\qedhere
    \end{align*}
\end{proof}

Now we can derive the definition of a geodesic.
\begin{defn}
    A curve $\gamma:[0,a]$ is called a \emph{geodesic} if $\nabla_{\dot\gamma(t)}\dot\gamma(t)=0$ for $t\in[0,a]$.
\end{defn}

\begin{rem}
    Geodesics are constant speed, this can be shown by $\frac{\d{}}{\d{t}}|\dot\gamma(t)|^2=2\langle\dot\gamma(t),\nabla_{\dot\gamma(t)}\dot\gamma(t)\rangle=0$.
\end{rem}

\begin{cor}\label{E'(0)=0}
    $\gamma$ is a critical value for all proper variation if and only if $\nabla_{\dot\gamma(t)}\dot\gamma(t)=0$, that is, $\gamma$ is a geodesic.
\end{cor}

We can give the proof of Proposition~\ref{energy-min to length-min}~now.
\begin{proof}[Proof of Proposition~\ref{energy-min to length-min}]
    Let $\gamma:[0,a]\to M$ be a curve such that for any $\tilde\gamma:[0,a]\to M$ with $\tilde\gamma(0)=\gamma(0)$, $\tilde\gamma(1)=\gamma(1)$, the inequality $E[\gamma]\leq E[\tilde\gamma]$ holds, we show that $L[\gamma]\leq L[\tilde\gamma]$.
    Let $\gamma(t,s)$ be any proper variation with $\gamma(t,0)=\gamma(t)$, then $\gamma$ is a critical point of $E(s)$.
    Hence by Corollary~\ref{E'(0)=0}, $\gamma$ is a geodesic.
    Now we can reparametrize $\tilde\gamma$ into arc-length, obtaining $\hat{\tilde\gamma}$.
    Therefore
    \[L^2[\gamma]=2aE[\gamma]\leq 2aE\left[\hat{\tilde\gamma}\right]=L^2\left[\hat{\tilde\gamma}\right]=L^2[\tilde\gamma],\]
    which implies $L[\gamma]\leq L[\tilde\gamma]$.
\end{proof}

Combining all results above, we have
\begin{prop}
    If a curve is length-minimizing, then it is a geodesic.
\end{prop}

\section{Geodesics and Exponential Maps}
To prove the local length-minimizing property of geodesic, we need to introduce the exponential map.

We first need to investigate the equation that determine a geodesic.
\begin{prop}
    Given $p\in M$ and $v\in T_pM$, there exists a unique geodesic $\gamma$ (whose domain may not be maximal) such that $\gamma(0)=p$, $\dot\gamma(0)=v$.
\end{prop}
\begin{proof}
    Let $(U,\varphi)$ be a local chart containing $p$, compose $\varphi$ with $\gamma$ we obtain coordinate curves $x^i$'s.
    Then the geodesic equation is equivalent to
    \[\ddot{x}^k(t)+\Gamma^k_{ij}(\gamma(t))\dot{x}^i(t)\dot{x}^j(t)=0,\ k=1,\cdots,n.\]
    This is a system of second order ordinary differential equations, by the unique existence theorem of ODE, the solution is completely determined by $x^i$'s and $\dot{x}^i$'s, that is, $p$ and $v$.
\end{proof}

Since the solution of an ODE relies continuously on its initial value, we have the following proposition.
\begin{prop}\label{geodesic existence}
    For any $p\in M$, there exists a neighborhood $V$ of $p$, such that there exists $\delta>0$, $\varepsilon>0$ and a smooth map $\gamma:(-\delta,\delta)\times\mathscr{U}\to M$, where $\mathscr{U}=\{(q,v)\in TM:\ q\in V,\ v\in T_qM,\ |v|<\varepsilon\}$, such that $\gamma(t;q,v)$ is a geodesic with $\gamma(0)=q$, $\dot\gamma(0)=v$.
\end{prop}
A proof can be found in~\cite[Chapter 3, Lemma 1]{Wu}.

Observe that $\gamma(\lambda t;p,v)=\gamma(t;p,\lambda v)$.
Denote $\gamma(t;p,v)$ by $\gamma_v(t)$, then above observation can be written as $\gamma_{\lambda v}(t)=\gamma_v(\lambda t)$.
Therefore, we can shorten the initial vector to lengthen the domain of geodesic.

\begin{defn}
    Let $U\subset T_pM$ be a neighborhood of origin, such that for any $v\in U$, $\gamma_v(1)$ is defined (existence is guaranteed by Proposition~\ref{geodesic existence}).
    We define the \emph{exponential map} at $p$ by
    \begin{align*}
        \exp_p:U&\to M\\
        v&\mapsto \gamma_v(1).
    \end{align*}
\end{defn}

\begin{rem}
    We can scale the initial vector and obtain
    \[\exp_p(v)=\gamma_v(1)=\gamma_{v/|v|}(|v|).\]
    This means the action of exponential map on $v$ is to move forward the distance $|v|$ along the geodesic with initial direction $v/|v|$.
\end{rem}

\begin{prop}\label{exp at 0}
    $\exp_{p*}|_0:T_0(T_pM)\to T_pM$ is identity (we identify $T_0(T_pM)$ with $T_pM$).
\end{prop}
\begin{proof}
    We have
    \[\exp_{p*}|_0(v)=\left.\frac{\d{}}{\d{t}}\right|_{t=0}(tv)=v.\qedhere\]
\end{proof}

\begin{cor}
    There exists a ball $B_\varepsilon(0)\subset T_pM$ such that $\exp_p:B_\varepsilon(0)\to M$ is a diffeomorphism onto its image.
\end{cor}
\begin{proof}
    Since $\exp_{p*}|_0$ is identity, it is nondegenerate, the corollary follows from inverse function theorem.
\end{proof}

Exponential maps enjoys naturality.
\begin{prop}\label{exp natural}
    Let $f:M\to N$ be a local isometry, $p\in M$, $q\in N$ and $f(p)=q$.
    Then we have the following commutative diagram
    \[\begin{tikzcd}
        U\subset T_pM \ar[r, "{f_{*p}}"] \ar[d, "{\exp_p}"] & V\subset T_qN \ar[d, "{\exp_q}"] \\
        M \ar[r, "f"] & N.
    \end{tikzcd}\]
\end{prop}
\begin{proof}
    This is simply because local isometry preserves geodesics.
\end{proof}

\begin{eg}
    \begin{enumerate}[(1)]
        \item We know that the geodesics on $\mathbb{S}^n$ are great circles, hence $\exp_p$ is defined on the whole $T_pM$.
        But $\exp_p$ is not injective, since 
        \[\exp_p(0)=\exp(2\pi v)=p\]
        for unit vector $v$ in $T_pM$.
        \item Let $M=\mathbb{S}^1\times\mathbb{R}$ be the cylinder.
        We know from elementary differential geometry that the geodesics on cylinder are directrix circles, helices and generatrix lines.
        Then in local charts $(e^{2\pi it},s)\mapsto(t,s)$, we know $\exp_p$ is not injective in the direction $(1,0)$, and injective in other directions.
    \end{enumerate}
\end{eg}

We postpone the discussion on whether the exponential map can be defined on the whole tangent space, the answer is Hopf--Rinow Theorem, which will be discussed in next section.

Now we prove that geodesics are locally length-minimizing.
For this, we introduce some local charts.
Given a Riemannian manifold $(M,g)$ and $p\in M$, let $\exp_p:B_\varepsilon(0)\to\exp(B_\varepsilon(0))=B_\varepsilon(p)$ be a diffeomorphism.

\begin{defn}
    We define \emph{geodesic normal coordinate} as follows:
    Let $\{e_i\}$ be an orthonormal basis of Euclidean space $(T_pM,\delta)$, $\{\alpha^i\}$ be its dual basis.
    Then we define the coordinate by
    \[q\in B_\varepsilon(p)\mapsto(\alpha^1(\exp_p^{-1}(q)),\cdots,\alpha^n(\exp_p^{-1}(q))).\]
\end{defn}

\begin{prop}
    Under geodesic normal coordinate, we have
    \[g_{ij}(p)=\delta_{ij},\ \Gamma^k_{ij}(p)=0.\]
\end{prop}
\begin{proof}
    Since $\exp_p$ is a diffeomorphism, we have $\left.\frac{\partial{}}{\partial{x^i}}\right|_p=\exp_{p*}|_0(e_i)=e_i$, hence $g_{ij}=\delta(e_i,e_j)=\delta_{ij}$.
    Moreover, let $x(t)=ty$ for $y\in T_pM-\{0\}$, then $x(t)$ is the coordinate of some geodesic in $B_\varepsilon(p)$, thus it satisfies the equation
    \[\ddot{x}^k(t)+\Gamma^k_{ij}(x(t))\dot{x}^i(t)\dot{x}^j(t)=0.\]
    Since $\dot{x}^i=y^i\neq 0$, $\ddot{x}^k=0$, we have $\Gamma^k_{ij}(ty)=0$.
    Let $y\to 0$ we obtain the conclusion.
\end{proof}

Next we introduce the geodesic polar coordinate.
\begin{defn}
    We define \emph{geodesic polar coordinate} as follows:
    Let $(r,\theta^1,\cdots,\theta^{n-1})$ be a polar coordinate on Euclidean space $(T_pM,\delta)$, then we defined the coordinate by
    \[q\in B_\varepsilon(p)-\{p\}\mapsto(r(\exp_p^{-1}(q)),\theta^1(\exp_p^{-1}(q)),\cdots,\theta^{n-1}(\exp_p^{-1}(q))).\]
\end{defn}

\begin{prop}
    Under geodesic polar coordinate, we have
    \[\left\langle\frac{\partial}{\partial{r}},\frac{\partial}{\partial{r}}\right\rangle=1,\ \left\langle\frac{\partial}{\partial{r}},\frac{\partial}{\partial{\theta^i}}\right\rangle=0.\]
\end{prop}
\begin{proof}
    To make things clear, we write the inverse of geodesic polar coordinate as
    \[F:(r,\omega)\mapsto\exp_p(r\omega)\]
    for $r\in(0,\varepsilon)$, $\omega\in\mathbb{S}^{n-1}$.
    Then we use $\partial_0,\partial_1,\cdots,\partial_{n-1}$ to denote the tangent vectors in $(0,\varepsilon)\times\mathbb{S}^{n-1}$, we have
    \begin{align*}
        &\frac{\partial{}}{\partial{r}}=F_*(\partial_0),\\
        &\frac{\partial{}}{\partial{\theta^i}}=F_*(\partial_i),\ i=1,\cdots,n-1.
    \end{align*}
    First we know that $\partial_0$ is the tangent vector of radial line $r\omega$, hence $\partial/\partial{r}$ is the tangent vector of a unit-speed radial geodesic, that is,
    \[\left\langle\frac{\partial{}}{\partial{r}},\frac{\partial{}}{\partial{r}}\right\rangle=1.\]
    Moreover, we have
    \begin{align*}
        \frac{\partial}{\partial{r}}\left\langle\frac{\partial{}}{\partial{r}},\frac{\partial{}}{\partial{\theta^i}}\right\rangle&=\left\langle\nabla_{\frac{\partial{}}{\partial{r}}}\frac{\partial{}}{\partial{r}},\frac{\partial{}}{\partial{\theta^i}}\right\rangle+\left\langle\frac{\partial{}}{\partial{r}},\nabla_{\frac{\partial{}}{\partial{r}}}\frac{\partial{}}{\partial{\theta^i}}\right\rangle\\
        &=\left\langle\frac{\partial{}}{\partial{r}},\nabla_{\frac{\partial{}}{\partial{r}}}\frac{\partial{}}{\partial{\theta^i}}\right\rangle\\
        &=\left\langle\frac{\partial{}}{\partial{r}},\nabla_{\frac{\partial{}}{\partial{\theta^i}}}\frac{\partial{}}{\partial{r}}\right\rangle\\
        &=\frac{1}{2}\frac{\partial{}}{\partial{\theta^i}}\left\langle\frac{\partial{}}{\partial{r}},\frac{\partial{}}{\partial{r}}\right\rangle\\
        &=0,
    \end{align*}
    hence $\left\langle\frac{\partial{}}{\partial{r}},\frac{\partial{}}{\partial{\theta^i}}\right\rangle$ is constant.
    However, if we let $r\to 0$, we have $\partial/\partial{\theta^i}\to 0$, therefore
    \[\left\langle\frac{\partial{}}{\partial{r}},\frac{\partial{}}{\partial{\theta^i}}\right\rangle=0.\]
\end{proof}

\begin{cor}
    Under geodesic polar coordinate, the metric tensor has local expression
    \[g=\d{r^2}+g_{ij}(r,\theta)\d{\theta^i}\otimes\d{\theta^j},\]
    where $[g_{ij}]_{i,j>0}$ is positive definite.
\end{cor}

As an application, we prove that geodesics are locally length-minimizing as we promised.

\begin{prop}\label{geodesic locally length-min}
    Let $\gamma:[0,1]\to M$ be a geodesic contained in an open set $U$, where geodesic polar coordinate is defined on $U$.
    Let $\tilde\gamma$ be any curve contained in $U$ with $\tilde\gamma(0)=\gamma(0)=p$, $\tilde\gamma(1)=\gamma(1)=q$.
    Then $L[\gamma]\leq L[\tilde\gamma]$.
\end{prop}
\begin{proof}
    Let $q=\exp_p(v)$, $\varphi$ is the geodesic polar coordinate.
    Then we have
    \[\gamma(t)=\varphi(tr_0,\omega_0),\ \tilde\gamma(t)=\varphi(r(t),\omega(t))\]
    such that $r(1)=r_0$, $\omega(t)\in\mathbb{S}^{n-1}$.
    Therefore
    \begin{align*}
        L[\gamma]&=\int_0^1|\dot\gamma(t)|\d{t}\\
        &=\int_0^1r_0\d{t}=r_0,\\
        L[\tilde\gamma]&=\int_0^1(|\dot{r}|^2(t)+g_{ij}\dot\theta^i(t)\dot\theta^j(t))^{1/2}\\
        &\geq\int_0^1|\dot{r}(t)|\d{t}\\
        &\geq\int_0^1\dot{r}(t)\d{t}=r_0.\qedhere
    \end{align*}
\end{proof}

\begin{rem}
    The hypothesis of Proposition~\ref{geodesic locally length-min}~can be weakened to $\exp_p$ is an immersion on $U$.
\end{rem}

\section{Hopf--Rinow Theorem}

We now answer the problem whether length-minimizing curve always exists.
The answer is Hopf--Rinow Theorem.

We adopt the metric geometry version of Hopf--Rinow Theorem from~\cite{Burago}, using geometric approach to prove the theorem and keeping the differential tools minimal.
We first define
\begin{defn}
    A Riemannian manifold $(M,g)$ is called \emph{(geodesically) complete} if $M$ is complete as a metric space.
\end{defn}

Our theorem is
\begin{thm}[Hopf--Rinow--Cohn--Vossen]\label{Hopf-Rinow}
    Let $(M,g)$ be a Riemannian manifold, the following four assertions are equivalent:
    \begin{enumerate}[(1)]
        \item $M$ has the Heine--Borel property, i.e.\ every closed geodesic ball is compact;
        \item $M$ is geodesically complete;
        \item Every geodesic $\gamma:[0,a)\to M$ can be extended to a continuous curve $\overline{\gamma}:[0,a]\to M$;
        \item There is a point $p\in M$ such that every length-minimizing geodesic $\gamma:[0,a)\to M$ with $\gamma(0)=p$ can be extended to a continuous curve $\overline{\gamma}:[0,a]\to M$.
    \end{enumerate}
\end{thm}

We first establish several lemmas.

\begin{lem}
    The length functional is lower semi-continuous in the following sense:
    let $\gamma_i,\gamma:[a,b]\to M$, if $\gamma_i\to\gamma$ pointwisely as $i\to\infty$, then
    \[L[\gamma]\leq\lim_{i\to\infty}L[\gamma_i].\]
\end{lem}
\begin{proof}
    Let $Y$ be a partition of $[a,b]$ with $a=y_0<\cdots<y_N=b$, denote
    \[\Sigma(Y)=\sum_{i=1}^Nd(\gamma_(y_{i-1}),\gamma_(y_i)).\]
    Take $\varepsilon>0$ and fix a partition $Z$ such that $L[\gamma]-\Sigma(Z)<\varepsilon$.
    Now consider $\Sigma_j(Z)$ for curves $\gamma_j$ corresponding to same partition $Z$.
    Choose $j$ so large that the inequality $d(\gamma_j(z_i),\gamma(z_i))<\varepsilon$ holds for all $z_i\in Z$.
    Then
    \[L[\gamma]\leq\Sigma(Z)+\varepsilon\leq\Sigma_j(Z)+\varepsilon+(N+1)\varepsilon\leq L[\gamma_j]+(N+2)\varepsilon.\]
    Since $\varepsilon$ is arbitrary, the lemma holds.
\end{proof}

\begin{lem}\label{shortest path}
    Let $\gamma_i$ be length-minimizing curves for $i=1,2,\cdots$, suppose $\gamma_i\to\gamma$ pointwisely, then $\gamma$ also minimizes length.
\end{lem}
\begin{proof}
    Let $\gamma$ has end points $p$ and $q$, then since $L[\gamma_i]$ equals to the distance between its end points, $L[\gamma_i]\to d(p,q)$.
    By the lower semi-continuity of length functional, we have
    \[L[\gamma]\leq\lim_{i\to\infty}L[\gamma_i]=d(p,q)\leq L[\gamma].\qedhere\]
\end{proof}

\begin{lem}\label{compact length-min}
    Let $(M,g)$ be a compact Riemannian manifold (maybe with boundary), then any two points $p,q\in M$ can be joined by a length-minimizing curve.
\end{lem}
\begin{proof}
    By the definition of distance, there exists a sequence of curves $\{\gamma_i\}$ with constant speed such that $\gamma_i(0)=p$, $\gamma_i(1)=q$ and $L[\gamma_i]\to d(p,q)$.
    Then let $L[\gamma_i]<A$ for all $i=1,2,\cdots$, we have $d(p,x)\leq L[\gamma_i|_{[0,a]}]<A$ for $x=\gamma_i(a)$, hence the family $\{\gamma_i\}$ is uniformly bounded.
    Moreover, fix $\varepsilon>0$, let $\delta=\varepsilon/A$, we have
    \[d(x,y)\leq L[\gamma_i|_{[a,b]}]\leq A(b-a)<\varepsilon\]
    provided $x=\gamma_i(a)$, $y=\gamma_i(b)$, and $b-a<\delta$ for any $i=1,2,\cdots$, hence the family $\{\gamma_i\}$ is equicontinuous.
    Then the family verifies the conditions of Arzela--Ascoli Theorem, it converges (up to a subsequence) to a curve $\gamma$.
    Then by the lower semi-continuity of length functional, we have
    \[L[\gamma]\leq\lim_{i\to\infty}L[\gamma_i]=d(p,q)\leq L[\gamma].\]
    Thus $\gamma$ is length-minimizing.
\end{proof}

\begin{proof}[Proof of Theorem~\ref{Hopf-Rinow}]
    Implications (1)$\implies$(2)$\implies$(3)$\implies$(4) are all easy, we prove (4)$\implies$(1).

    Let $R=\sup\{\overline{B_r(p)}\ \text{is a compact set}\}$, then $R>0$ since manifolds are locally compact, $\overline{B_r(p)}$ is compact for $r$ sufficiently small.
    We argue by contradiction.
    Suppose $R<+\infty$, that is, there exists noncompact geodesic balls.
    The argument is divided into two steps.

    1. First we prove $B_R(p)$ is sequentially compact.
    Let $\{p_i\}\subset B_R(p)$, set $d(p,p_i)=r_i$.
    We may assume $r_i\to R$ as $i\to\infty$, otherwise $\{p_i\}$ is eventually contained in a smaller geodesic ball, and it has a convergent subsequence by the definition of $R$.

    Now let $\gamma_i:[0,r_i]\to M$ be a length-minimizing curve joining $p$ and $p_i$, whose existence is guaranteed by Lemma~\ref{compact length-min}.
    Notice that $\gamma_i$'s are parametrized by arc-length.
    We can choose a subsequence of $\{\gamma_i\}$ such that the restrictions of the curves to $[0,r_1]$ converge by Arzela--Ascoli Theorem.
    From this subsequence, we can choose a further subsequence such that the restrictions to $[0,r_2]$ converge, and so on.
    Then by Cantor diagonal procedure, we have a sequence $\{\gamma_{i_n}\}$ such that for $t\in[0,R)$, $\gamma_{i_n}(t)$ is well-defined for $n$ sufficiently large and $\gamma_{i_n}(t)\to\gamma(t)$ as $n\to\infty$.
    Moreover, Arzela--Ascoli Theorem asserts $\gamma$ is smooth for any restriction to $[0,r]$ provided $r<R$ by uniform convergence, hence $\gamma$ is smooth.

    Now by Lemma~\ref{shortest path}, $\gamma$ is a length-minimizing curve, hence by the hypothesis (iv), $\gamma$ can be extended to a continuous curve $\overline{\gamma}:[0,R]\to M$.
    Then let $q=\overline{\gamma}(R)$, fix $\varepsilon>0$, take $n$ sufficiently large such that $d(p_{i_n},\gamma_{r_{i_n}})<\varepsilon/2$ and $d(\gamma(r_{i_n},q))=R-r_{i_n}<\varepsilon/2$, we have
    \[d(p_{i_n},q)\leq d(p_{i_n},\gamma(r_{i_n}))+d(\gamma(r_{i_n}),q)<\varepsilon.\]
    Hence $p_{i_n}\to q$ as $n\to\infty$, we have $B_R(p)$ is sequentially compact.

    2. Since $B_R(p)$ is sequentially compact, we have $\overline{B_R(p)}$ is compact.
    Now we show $\overline{B_{R+\varepsilon}(p)}$ is compact for some $\varepsilon>0$.
    Since $M$ is locally compact, for every $x\in\overline{B_R(0)}$ there is an $r(x)>0$ such that $\overline{B_{r(x)}(x)}$ is compact.
    Then we can choose finite $x_i\in\overline{B_R(p)}$ such that $\{B_{r(x_i)}(x_i)\}$ covers $\overline{B_R(p)}$.
    The union of these geodesic balls is sequentially compact and contains the geodesic ball $B_{R+\varepsilon}(p)$ for $0<\varepsilon<\min\{r(x_i)\}$.
    Hence $\overline{B_{R+\varepsilon}(p)}$ is compact, this contradicts the choice of $R$.
\end{proof}

We have two corollaries.

\begin{cor}[Hopf--Rinow Theorem]
    Let $(M,g)$ be a Riemannian manifold, the following are equivalent:
    \begin{enumerate}[(1)]
        \item $M$ is geodesically complete;
        \item $\exp_p$ is defined on whole $T_pM$ for any $p\in M$;
        \item $\exp_p$ is defined on whole $T_pM$ for one $p\in M$.
    \end{enumerate}
\end{cor}
\begin{proof}
    (1)$\implies$(2): Let $S=\{r\in\mathbb{R}_{>0}:\ \gamma:[0,r)\ \text{can be extended to }r\}$.
    We prove $S$ is both open and closed, then the implication holds.
    Let $r\in S$, for any $q\in\partial B_r(p)$, define $r(q)$ as follows:
    let unit-speed geodesic $\gamma:[0,r]\to M$ joins $p$ and $q$, then there exists a geodesic $\tilde\gamma:[r,r+r(q))$ such that $\tilde\gamma(r)=q$, $\dot{\tilde\gamma}(r)=\dot\gamma(r)$, that is, $\gamma$ can be extended to $[0,r+r(q))$.
    Cover $\partial B_r(p)$ by finite many $B_{r(q_i)}(q_i)$, then for $0<\varepsilon<\min\{r(q_i)\}$, $B_{r+\varepsilon}(p)$ is in the union of $B_r(p)$ and $B_{r(q_i)}(q_i)$, hence any geodesics $\sigma:[0,r+\varepsilon)$ can be extended to $\sigma:[0,r+\varepsilon]\to M$.
    This implies $(r-\varepsilon,r+\varepsilon)\subset S$, $S$ is open.

    Now assume $\{r_i\}\subset S$ converges to $r$.
    Let $\gamma:[0,r)$ be a geodesic, then define $\gamma(r)=\lim_{i\to\infty}\gamma(r_i)$, the limit exists since $M$ is complete.
    Hence $r\in S$, $S$ is closed.

    (2)$\implies$(3) is trivial.

    (3)$\implies$(1): If $\gamma:[0,a)$ is a length-minimizing geodesic with $\gamma(0)=p$, then $\gamma(t)=\exp_p(tv)$ for some $v$, and clearly it can be extended to $a$.
\end{proof}

\begin{cor}\label{join points}
    If $M$ is geodesically complete, then any two points can be joined by a length-minimizing geodesic.
\end{cor}
\begin{proof}
    Let $p,q\in M$, $R>d(p,q)$.
    Then $q\in\overline{B_R(p)}$, the latter one is bounded and closed, hence by Theorem~\ref{Hopf-Rinow}, it is compact.
    By Lemma~\ref{compact length-min}, there is a length-minimizing geodesic joining $p$ and $q$.
\end{proof}
\chapter{Curvature}

\section{Curvature Tensor and Curvaure Endomorphism}\label{curvature section}

\subsection*{Riemann Curvature Tensor}
We calculated in Proposition~\ref{exp at 0}~that $\exp_{p*}|_0$ is identity, naturally we have the following problem:
\begin{pro}
    Calculate $\exp_{p*}|_v:T_v(T_pM)\to T_{\exp_p(v)}M$.
\end{pro}
\begin{proof}[Solution]
    To evaluate $\exp_{p*}|_v(\xi)$, we choose a line $v+s\xi$, and then
    \[\exp_{p*}|_v(\xi)=\left.\frac{\d{}}{\d{s}}\right|_{s=0}\exp_p(v+s\xi).\]
    Now we can introduce the one parameter family of geodesics 
    \[\gamma(t,s)=\exp_p(t(v+s\xi)),\]
    and denote $\gamma(t)=\gamma(t,0)$.
    We calculate the variation vector field $J(t)$ of $\gamma$, and obtain the result by taking $t=1$.
    Let $J_s(t)=\frac{\partial{}}{\partial{s}}\gamma(t,s)$, then $\dot{J}_s(t)=\nabla_{\dot\gamma_s(t)}\frac{\partial\gamma}{\partial{s}}$.
    Since $\nabla_{\dot\gamma_s(t)}\frac{\partial\gamma}{\partial{t}}=0$, we have
    \begin{align*}
        \ddot{J}_s(t)&=\nabla_{\dot\gamma_s(t)}\nabla_{\dot\gamma_s(t)}\frac{\partial\gamma}{\partial{s}}\\
        &=\nabla_{\dot\gamma_s(t)}\nabla_{\frac{\partial\gamma}{\partial{s}}}\frac{\partial\gamma}{\partial{t}}\\
        &=\nabla_{\frac{\partial\gamma}{\partial{t}}}\nabla_{\frac{\partial\gamma}{\partial{s}}}\frac{\partial\gamma}{\partial{t}}-\nabla_{\frac{\partial\gamma}{\partial s}}\nabla_{\frac{\partial\gamma}{\partial{t}}}\frac{\partial\gamma}{\partial{t}}.
    \end{align*}
    Moreover, we have $[\partial_t,\partial_s]=0$, then we denote
    \[R\left(\frac{\partial}{\partial{t}},\frac{\partial}{\partial{s}}\right)=\nabla_{\frac{\partial}{\partial{s}}}\nabla_{\frac{\partial}{\partial{t}}}-\nabla_{\frac{\partial}{\partial{t}}}\nabla_{\frac{\partial}{\partial{s}}}+\nabla_{\left[\frac{\partial}{\partial{t}},\frac{\partial}{\partial{s}}\right]},\]
    we obtain
    \[\ddot{J}_s(t)+R\left(\frac{\partial\gamma}{\partial{t}},\frac{\partial\gamma}{\partial{s}}\right)\frac{\partial\gamma}{\partial{t}}=0.\]
    Take $s=0$, we have
    \begin{equation}
        \ddot{J}(t)+R(\dot\gamma(t),J(t))\dot\gamma(t)=0.\label{Jacobi field}
    \end{equation}
    We will show later that~\eqref{Jacobi field}~is a system of ordinary differential equations, hence by solving the system with given initial value and taking $t=1$, we obtain the answer of the problem.
\end{proof}

We make it into a definition.

\begin{defn}
    The \emph{Riemann curvature tensor} $R:\Gamma(TM)\times\Gamma(TM)\times\Gamma(TM)\to\Gamma(TM)$ is defined by
    \[R(X,Y)Z=\nabla_Y\nabla_XZ-\nabla_X\nabla_YZ+\nabla_{[X,Y]}Z.\]
\end{defn}

\begin{rem}
    Many authors define the Riemann curvature tensor as the negative of above definition, e.g.\ in~\cite{Petersen}.
    Please be careful with the sign of the tensor.
\end{rem}

We need to explain the name ``tensor'', so we must show $R$ is truly tensorial.
\begin{lem}\label{R tensor}
    $R$ is a tensor.
\end{lem}
\begin{proof}
    $R(X,Y)Z$ is clearly tensorial in $X$ and $Y$, we show that $R(X,Y)(fZ)=fR(X,Y)Z$ for $f\in C^\infty(M)$.
    We have
    \begin{align*}
        \nabla_Y\nabla_X(fZ)&=\nabla_Y((Xf)Z+f\nabla_XZ)\\
        &=(YXf)Z+(Xf)\nabla_YZ+(Yf)\nabla_XZ+f\nabla_Y\nabla_XZ\\
        -\nabla_X\nabla_Y(fZ)&=-(XYf)Z-(Yf)\nabla_XZ-(Xf)\nabla_YZ-f\nabla_X\nabla_YZ\\
        \nabla_{[X,Y]}(fZ)&=([X,Y]f)Z+f\nabla_{{X,Y}}Z,
    \end{align*}
    add all three equalities and we reach the conclusion.
\end{proof}

Now we can look at Riemann curvature tensor locally.
A rather complicated calculation shows:
\begin{lem}
    Let $R=\R{l}{i}{j}{k}\otimes\d{x^i}\otimes\d{x^j}\otimes\d{x^k}\otimes\partial_l$, then
    \[\R{l}{i}{j}{k}=\partial_j\Gamma^l_{ik}-\partial_i\Gamma^l_{jk}+\Gamma^m_{ik}\Gamma^l_{jm}-\Gamma^m_{jk}\Gamma^l_{im}.\]
\end{lem}

We also define a $(0,4)$-tensor by lowering the $l$ index of $R{l}{i}{j}{k}$, that is:
\begin{defn}
    $R(X,Y,Z,W):=\langle R(X,Y)Z,W\rangle$ is also called \emph{Riemann curvature tensor}.
\end{defn}

\begin{eg}
    Euclidean space $(\mathbb{R}^n,\delta)$ has $R\equiv 0$.
    Any metric admits zero curvature is called \emph{flat}.
\end{eg}

\subsection*{Curvature Endomorphism}
Curvature also appears in another scene.
Let us consider the second covariant differential of a tensor.

\begin{defn}
    Let $T(\cdots)$ be a tensor, denote $(\nabla_{X,Y}T)(\cdots):=\nabla(\nabla T)(\cdots,Y,X)$.
\end{defn}

\begin{prop}
    For $(r,s)$-tensor $T$, we have
    \begin{equation}
        \nabla_{X,Y}T=\nabla_X(\nabla_YT)-\nabla_{\nabla_XY}T.\label{2nd cov deriv}
    \end{equation}
\end{prop}
\begin{proof}
    Since covariant derivative commutes with contraction, we have
    \begin{align*}
        \nabla_X(\nabla_YT)&=\nabla_X(\tr_{1,s+2}Y\otimes\nabla T)\\
        &=\tr_{1,s+2}(\nabla_X(Y\otimes\nabla T))\\
        &=\tr_{1,s+2}((\nabla_XY)\otimes\nabla T+Y\otimes(\nabla_X\nabla T))\\
        &=\nabla_{\nabla_XY}T+\nabla(\nabla T)(\cdots,Y,X)\\
        &=\nabla_{\nabla_XY}T+\nabla_{X,Y}T,
    \end{align*}
    then the result follows.
\end{proof}

\begin{defn}
    Define the \emph{curvature endomorphism} $R(X,Y)$ on $(r,s)$-tensors by
    \[R(X,Y)T=\nabla_Y\nabla_XT-\nabla_X\nabla_YT+\nabla_{[X,Y]}T.\]
\end{defn}

\begin{rem}
    We need to show $R(X,Y)T$ is tensorial in $T$ so that $R(X,Y)$ is a well-defined endomorphism.
    This is similar to Lemma~\ref{R tensor}.
\end{rem}

\begin{prop}\label{Ricci identity}
    For any $(r,s)$-tensor $T$, we have the following \emph{Ricci identity}:
    \[\nabla_{Y,X}T-\nabla_{X,Y}T=R(X,Y)T.\]
    Moreover, we have a explicit formula
    \begin{equation}
        \begin{aligned}
            &(R(X,Y)T)(\omega_1,\cdots,\omega_r,X_1,\cdots,X_s)\\
            =&-\sum_{i=1}^rT(\omega_1,\cdots,R(X,Y)\omega_i,\cdots,\omega_r,X_1,\cdots,X_s)\\
            &-\sum_{j=1}^sT(\omega_1,\cdots,\omega_r,X_1,\cdots,R(X,Y)X_j,\cdots,X_s)
        \end{aligned}\label{curvature formula}
    \end{equation}
\end{prop}
\begin{proof}
    Using equation~\eqref{2nd cov deriv}, we have
    \begin{align*}
        \nabla_{Y,X}T-\nabla_{X,Y}T&=\nabla_Y\nabla_XT-\nabla_{\nabla_YX}T-\nabla_X\nabla_YT+\nabla_{\nabla_XY}T\\
        &=\nabla_Y\nabla_XT-\nabla_X\nabla_YT+\nabla_{[X,Y]}T\\
        &=R(X,Y)T,
    \end{align*}
    the second equality is torsion-freeness.
    Since $R(X,Y)$ clearly satisfies Leibniz Law and commutes with contraction, we can deduce the formula in the same way as covariant derivative.
\end{proof}

Ricci identity shows the curvature appears when we interchange the second covariant differential.

\subsection*{Properties of Curvature}
\begin{prop}\label{curvature symmetric}
    Riemann curvature tensor has following symmetric properties:
    For $X,Y,Z,W\in\Gamma(TM)$, we have
    \begin{enumerate}[(1)]
        \item $R(X,Y,Z,W)=-R(Y,X,Z,W)=-R(X,Y,W,Z)$;
        \item $R(X,Y)Z+R(Y,Z)X+R(Z,X)Y=0$ (First Bianchi identity);
        \item $R(X,Y,Z,W)=R(Z,W,X,Y)$;
        \item $(\nabla_XR)(Y,Z)+(\nabla_YR)(Z,X)+(\nabla_ZR)(X,Y)=0$ (Second Bianchi identity).
    \end{enumerate}
\end{prop}
\begin{proof}
    (1) The first equality is evident.
    We show the second equality.
    Consider the Hessian of $g(Z,W)$, then we have
    \[\nabla^2g(Z,W)(X,Y)=\langle\nabla^2_{Y,X}Z,W\rangle+\langle\nabla_XZ,\nabla_YW\rangle+\langle\nabla_YZ,\nabla_XW\rangle+\langle Z,\nabla^2_{Y,X}W\rangle.\]
    Interchange $X,Y$, we have
    \[\nabla^2g(Z,W)(Y,X)=\langle\nabla^2_{X,Y}Z,W\rangle+\langle\nabla_YZ,\nabla_XW\rangle+\langle\nabla_XZ,\nabla_YW\rangle+\langle Z,\nabla^2_{X,Y}W\rangle.\]
    These two equations must equal, hence
    \[\langle\nabla^2_{Y,X}Z,W\rangle-\langle\nabla^2_{X,Y}Z,W\rangle=\langle Z,\nabla^2_{X,Y}W\rangle-\langle Z,\nabla^2_{Y,X}W\rangle,\]
    this is equivalent to
    \[R(X,Y,Z,W)=R(Y,X,W,Z).\]
    (2) Since $R(X,Y)Z$ is tensorial, we can assume $X,Y,Z$ are frames.
    Then all Lie bracket between $X,Y,Z$ vanish, we have
    \begin{align*}
        \sum_{\text{cyc}}R(X,Y)Z=&\nabla_Y\nabla_XZ-\nabla_X\nabla_YZ+\nabla_Z\nabla_YX-\nabla_Y\nabla_ZX\\
        &+\nabla_X\nabla_ZY-\nabla_Z\nabla_XY\\
        =&\nabla_Y[X,Z]+\nabla_Z[Y,X]+\nabla_X[Z,Y]\\
        =&0.
    \end{align*}
    (3) By (1) and (2) we have
    \begin{align*}
        R(X,Y,Z,W)=&-R(Z,X,Y,W)-R(Y,Z,X,W)\\
        =&R(Z,X,W,Y)+R(Y,Z,W,X)\\
        =&-R(W,Z,X,Y)-R(X,W,Z,Y)\\
        &-R(W,Y,Z,X)-R(Z,W,Y,X)\\
        =&2R(Z,W,X,Y)+R(X,W,Y,Z)+R(W,Y,X,Z)\\
        =&2R(Z,W,X,Y)-R(Y,X,W,Z)\\
        =&2R(Z,W,X,Y)-R(X,Y,Z,W),
    \end{align*}
    which implies $2R(X,Y,Z,W)=2R(Z,W,X,Y)$.\\
    (4) By the definition of covariant derivative, we have
    \begin{align*}
        (\nabla_XR)(Y,Z)=&[\nabla_X,R(Y,Z)]-R(\nabla_XY,Z)-R(Y,\nabla_XZ)\\
        =&[\nabla_X,\nabla_{[Y,Z]}]-[\nabla_X,[\nabla_Y,\nabla_Z]]-R(\nabla_XY,Z)-R(Y,\nabla_XZ)\\
        =&\nabla_{[X,[Y,Z]]}-[\nabla_X,[\nabla_Y,\nabla_Z]]-R(X,[Y,Z])\\
        &-R(\nabla_XY,Z)-R(Y,\nabla_XZ).
    \end{align*}
    Take summation cyclically, we see that terms involving $\nabla$ vanish because of Jacobi identity.
    Moreover, since $\nabla_XY-\nabla_YX=[X,Y]$, the terms involving $R$ also vanish.
\end{proof}

\section{Sectional, Ricci and Scalar Curvature}

We now define some special curvatures.
Fix a Riemannian manifold $(M,g)$.

\begin{defn}
    Let $p\in M$, $\pi\subset T_pM$ be a $2$-plane, $\pi=\Span\{u,v\}$.
    Then define the \emph{sectional curvature} of $\pi$ at $p$ to be
    \[\Sect_p(\pi)=\frac{R_p(u,v,u,v)}{|u|^2|v|^2-\langle u,v\rangle^2}.\]
\end{defn}

\begin{rem}
    One can show that sectional curvature does not depend on the choice of basis.
    For a proof, see~\cite[Proposition~3.1]{doCarmo}.
\end{rem}

\begin{prop}\label{sect to curvature}
    The sectional curvature determines the curvature tensor.
\end{prop}
\begin{proof}
    Let $R,R'$ have same sectional curvature, denote $\tilde{R}=R-R'$, then $\widetilde{\Sect}=0$.
    We show that $\tilde{R}=0$.
    First, we have
    \begin{align*}
        \tilde{R}(X,Y,X,W)&=\tilde{R}(X,Y-W,X,W)+\tilde{R}(X,W,X,W)\\
        &=\tilde{R}(X,Y-W,X,W-Y)+\tilde{R}(X,Y-W,X,Y)\\
        &=\tilde{R}(X,Y,X,Y)-\tilde{R}(X,W,X,Y)\\
        &=\tilde{R}(X,Y,X,W),
    \end{align*}
    hence $\tilde{R}(X,Y,X,W)=0$.
    Then we have
    \begin{align*}
        \tilde{R}(X,Y,Z,W)&=\tilde{R}(X,Y,Z,W-X)+\tilde{R}(X,Y,Z,X)\\
        &=\tilde{R}(X-W,Y,Z,W-X)+\tilde{R}(W,Y,Z,W-X)\\
        &=\tilde{R}(W,Y,Z,W)-\tilde{R}(W,Y,Z,X)\\
        &=-\tilde{R}(W,Y,Z,X).
    \end{align*}
    By the same reasoning, we have
    \begin{align*}
        \tilde{R}(X,Y,Z,W)&=\tilde{R}(Z,W,X,Y)\\
        &=-\tilde{R}(Y,W,X,Z)\\
        &=-\tilde{R}(X,Z,Y,W).
    \end{align*}
    Passing the result to the $(1,3)$-tensor, by first Bianchi identity, we have
    \begin{align*}
        \tilde{R}(X,Z)Y+\tilde{R}(Y,Z)X+\tilde{R}(Z,X)Y=0,
    \end{align*}
    which implies $\tilde{R}(Y,Z)X=0$.
    Then
    \begin{align*}
        0&=\tilde{R}(Y,Z,X,W)\\
        &=-\tilde{R}(W,Z,X,Y)\\
        &=-\tilde{R}(X,Y,W,Z)\\
        &=\tilde{R}(X,Y,Z,W).
    \end{align*}
    Thus we proved $\tilde{R}=0$, and the result follows.
\end{proof}

We mention here a little observation.
\begin{lem}\label{curvature of const sect}
    If $\Sect_p$ is constant for all $2$-planes in $T_pM$, say $K_p$, then we have
    \[R_p(X,Y,Z,W)=K_p(\langle X,Z\rangle\langle Y,W\rangle-\langle X,W\rangle\langle Y,Z\rangle).\]
\end{lem}

Then we can prove the following Schur's Theorem.
\begin{thm}[Schur]
    Let $(M^n,g)$ be a Riemannian manifold with $n\geq 3$.
    If $\Sect_p(\pi)$ is independent from $\pi\subset T_pM$ for all $p\in M$, then $M$ has constant sectional curvature.
\end{thm}
\begin{proof}
    Since the tensor
    \[R'(X,Y,Z,W)=\langle X,Z\rangle\langle Y,W\rangle-\langle X,W\rangle\langle Y,Z\rangle\]
    evidently satisfies the Proposition~\ref{curvature symmetric}, Proposition~\ref{sect to curvature}~shows $R=fR'$ for some $f\in C^\infty(M)$.
    We show that $f$ is constant.
    Since $n\geq 3$, there exists three orthonormal vectors $X,Y,Z$.
    Take $W$ arbitrary, then by second Bianchi identity, we have
    \[(\nabla_XR)(Y,Z,W)+(\nabla_YR)(Z,X,W)+(\nabla_ZR)(X,Y,W)=0.\]
    Since $\nabla_XR=\nabla_X(fR')=(Xf)R'+f(\nabla_XR')$, by taking summation cyclically we obtain
    \[(Xf)R'(Y,Z)W+(Yf)R'(Z,X)W+(Zf)R'(X,Y)W=0.\]
    Since the sectional curvature is constant for any $\pi\subset T_pM$, the ``sectional curvature'' corresponding to $R'$ is also constant, hence we can use Lemma~\ref{curvature of const sect}~to obtain
    \begin{align*}
        0=&(Xf)K(\langle Y,W\rangle Z-\langle Z,W\rangle Y)\\
        &+(Yf)K(\langle Z,W\rangle X-\langle X,W\rangle Z)\\
        &+(Zf)K(\langle X,W\rangle Y-\langle Y,W\rangle X),
    \end{align*}
    which is equivalent to
    \begin{align*}
        0=&((Xf)\langle Y,W\rangle-(Yf)\langle X,W\rangle)Z\\
        &+((Yf)\langle Z,W\rangle-(Zf)\langle Y,W\rangle)X\\
        &+((Zf)\langle X,W\rangle-(Xf)\langle Z,W\rangle)Y.
    \end{align*}
    Since $X,Y,Z$ are orthonormal, the coefficient of $X,Y,Z$ must all equal to $0$.
    Thus by taking $W=Y$, we obtain
    \[Xf=(Xf)\langle Y,Y\rangle=(Yf)\langle X,Y\rangle=0.\]
    Since $X$ is arbitrary, we have $f\equiv\text{const}$.
    This deduces $M$ has constant sectional curvature.
\end{proof}

\begin{defn}
    Let $\R{l}{i}{j}{k}$ be the local expression of Riemann curvature tensor, then we define the \emph{Ricci curvature} by
    \[\Ricij{l}{j}=\tr_{13}\R{l}{i}{j}{k}.\]
    Equivalently, let $\{e_i\}$ be an orthonormal basis at $p$, then
    \[\Ric_p(X)=\sum_{i=1}^nR(e_i,X)e_i.\]
\end{defn}

Clearly, $\Ric$ is a self-adjoint linear transformation by Proposition~\ref{curvature symmetric}~(3).

\begin{defn}
    We define the \emph{scalar curvature} by taking trace of $\Ric$, or equivalently
    \[\Scal(p)=\sum_{i=1}^n\langle\Ric_p(e_i),e_i\rangle\]
    for an orthonormal basis $\{e_i\}$ at $p$.
\end{defn}

\begin{defn}
    A Riemannian manifold $(M,g)$ is called an \emph{Einstein manifold} if $\Ric=\lambda\delta$ for some $\lambda\in C^\infty(M)$.
\end{defn}

We have another theorem of Schur on Einstein manifolds.
\begin{thm}[Schur]\label{schur for scal}
    Let $M^n$ be an Einstein manifold with $n\geq 3$, then $M$ has constant scalar curvature.
\end{thm}

First we need a lemma.
\begin{lem}
    For any metric $g$, we have
    \[2\tr_{13}\nabla\Ric=\d\Scal.\]
\end{lem}
\begin{proof}
    We check the equation locally.
    The second Bianchi identity can be written as
    \[\R{l}{i}{j}{m}{}_{;k}+\R{l}{k}{i}{m}{}_{;j}+\R{l}{j}{k}{m}{}_{;i}=0,\]
    or equivalently
    \[\R{l}{i}{j}{m}{}_{;k}-\R{l}{i}{k}{m}{}_{;j}+\R{l}{j}{k}{m}{}_{;i}=0.\]
    Multiply $g^{im}$ we obtain
    \[0=\Ricij{l}{j}{}_{;k}-\Ricij{l}{k}{}_{;j}+R_{jk}{}^{il}{}_{;i}.\]
    Let $l=j$, we obtain
    \begin{align*}
        0&=\Ricij{j}{j}{}_{;k}-\Ricij{j}{k}{}_{;j}-R_{kj}{}^{ij}{}_{;i}\\
        &=\Scal_{;k}-\Ricij{j}{k}{}_{;j}-\Ricij{i}{k}{}_{;i}\\
        &=\Scal_{;k}-2\Ricij{i}{k}{}_{;i}.
    \end{align*}
    Then the result follows.
\end{proof}

\begin{proof}[Proof of Theorem~\ref{schur for scal}]
    Let $\Ric=\lambda\delta$, then by the lemma,
    \begin{align*}
        \d\Scal&=2\tr_{1,3}\nabla\Ric\\
        &=2\tr_{1,3}\nabla(\lambda\delta)\\
        &=2\tr_{1,3}(\delta\otimes\d{\lambda})\\
        &=2\d{\lambda}.
    \end{align*}
    However, we have
    \begin{align*}
        \d\Scal&=\d\tr_{1,2}\Ric\\
        &=\d{\lambda\tr_{1,2}\delta}\\
        &=n\d{\lambda}.
    \end{align*}
    This means $(n-2)\d\Scal=0$, which implies $\Scal\equiv\text{const}$ for $n\geq 3$.
\end{proof}

\section{Jacobi Fields}
At the beginning of Section~\ref{curvature section}, we introduced the one parameter family of geodesics $\gamma(t,s)=\exp_p(t(v+s\xi))$.
It is a variation of curve $\gamma(t)=\gamma(t,0)$, and its variation vector field $J$ satisfies~\eqref{Jacobi field}.
Motivated by this, we give the definition of Jacobi fields.

\begin{defn}
    Let $\gamma$ be a geodesic, a vector field $J$ along $\gamma$ is called a \emph{Jacobi field}, if $\ddot{J}(t)+R(\dot\gamma(t),J(t))\dot\gamma(t)=0$.
\end{defn}

Let $\{e_i(t)\}$ be a parallel frame along $\gamma$, and $J(t)=f^ie_i$.
Define $a^i_j(t)=R(\dot\gamma(t),e_i(t),\dot\gamma(t),e_j(t))$, then the equation of Jacobi field is equivalent to
\[\ddot{f}^i(t)+a^i_j(t)f^j(t)=0,\ i=1,\cdots,n.\]
Hence~\eqref{Jacobi field}~is indeed a system of ODE.
By ODE theory, given $f^i(0),\dot{f}^i(0)$ $i=1,\cdots,n$, the $f^i(t)$'s are uniquely determined.
Translate into geometric language, a Jacobi field $J$ is determined by $J(0)$ and $\dot{J}(0)$.

We summarize above discussion.

\begin{prop}
    Let $\mathscr{J}\gamma$ be the vector space of all Jacobi fields along $\gamma$, then 
    \[\dim\mathscr{J}(\gamma)=2n.\]
\end{prop}

The proposition below shows only Jacobi fields that perpendicular to $\gamma$ are interesting.
\begin{prop}
    Let $\gamma$ be a geodesic, $J$ is a Jacobi field along $\gamma$.
    Then we have the decomposition
    \[J(t)=J^\perp(t)+(at+b)\dot\gamma(t),\]
    where $J^\perp(t)\perp\dot\gamma(t)$.
    If a Jacobi field is perpendicular to $\gamma$, then we call it a \emph{normal Jacobi field}.
\end{prop}
\begin{proof}
    We have
    \begin{align*}
        \frac{\d^2{}}{\d{t^2}}\langle J(t),\dot\gamma(t)\rangle&=\langle\ddot{J}(t),\dot\gamma(t)\rangle\\
        &=-\langle R(\dot\gamma(t),J(t))\dot\gamma(t),\dot\gamma(t)\rangle\\
        &=0.\qedhere
    \end{align*}
\end{proof}

At the beginning of Section~\ref{curvature section}, we see that a one parameter family of geode\-sics gives rise to a Jacobi field.
The converse is also true.
\begin{prop}
    Let $J$ be a Jacobi field, then $J$ is the variation field of some one parameter family of geodesics.
\end{prop}
\begin{proof}
    Given a Jacobi field $J$, it is determined by $J(0)$ and $\dot{J}(0)$.
    Let $\zeta(s)$ be the geodesic with initial tangent vector $J(0)$, and $T(s)$ be its tangent vector field.
    Let $W(s)$ be parallel along $\zeta$ and $W(0)=\dot{J}(0)$.
    Now set $\gamma(t,s)=\exp_{\zeta(s)}(t(T(s)+sW(s)))$.
    Then
    \[\left[\frac{\partial}{\partial{t}},\frac{\partial}{\partial{s}}\right]=\gamma_*[\partial_t,\partial_s]=0,\]
    hence we can interchange the partial derivative.
    Let $U$ be the variation field of $\gamma(t,s)$, then $U$ is a Jacobi field.
    We verify $U$ and $J$ has same initial value.
    We have
    \begin{align*}
        U(0)&=\left.\frac{\partial{}}{\partial{s}}\right|_{s=0}\gamma(0,s)\\
        &=\left.\frac{\partial{}}{\partial{s}}\right|_{s=0}\exp_{\zeta(s)}(0)\\
        &=\left.\frac{\partial{}}{\partial{s}}\right|_{s=0}\zeta(s)\\
        &=\dot\zeta(0)=J(0),
    \end{align*}
    and
    \begin{align*}
        \dot{U}(0)&=\left.\left.\frac{\partial}{\partial{t}}\right|_{t=0}\frac{\partial}{\partial{s}}\right|_{s=0}\exp_{\zeta(s)}(t(T(s)+sW(s)))\\
        &=\left.\left.\frac{\partial}{\partial{s}}\right|_{t=0}\frac{\partial}{\partial{t}}\right|_{s=0}\exp_{\zeta(s)}(t(T(s)+sW(s)))\\
        &=\left.\frac{\partial}{\partial{s}}\right|_{s=0}\exp_{\zeta{s}*}|_0(T(s)+sW(s))\\
        &=\left.\frac{\partial}{\partial{s}}\right|_{s=0}(T(s)+sW(s))\\
        &=W(0)=\dot{J}(0).\qedhere
    \end{align*}
\end{proof}

Now we can completely answer the calculation problem of differential of exponential map.
\begin{prop}\label{compute exp*}
    Let a Jacobi field $J$ along $\gamma(t)=\exp_p(tv)$ satisfy $J(0)=0$, $\dot{J}(0)=\xi$, then $J(t)=\exp_{p*}|_{tv}(t\xi)$.
\end{prop}
\begin{proof}
    Let $J$ be the variation vector field of $\gamma(t,s)=\exp_p(t(v+s\xi))$.
    Then
    \begin{align*}
        J(t)&=\frac{\partial{}}{\partial{s}}\exp_p(t(v+s\xi))\\
        &=\exp_{p*}|_{tv}(t\xi).
    \end{align*}
    Now consider the initial condition for $J$, we have
    \[J(0)=\exp_{p*}|_0(0)=0,\]
    and
    \begin{align*}
        \dot{J}(0)&=\left.\frac{\partial}{\partial{t}}\right|_{t=0}\left.\frac{\partial{}}{\partial{s}}\right|_{s=0}\exp_p(t(v+s\xi))\\
        &=\dot{J}(0)=\left.\frac{\partial}{\partial{s}}\right|_{s=0}\left.\frac{\partial{}}{\partial{t}}\right|_{t=0}\exp_p(t(v+s\xi))\\
        &=\left.\frac{\partial{}}{\partial{s}}\right|_{s=0}\exp_{p*}|_0(v+s\xi)\\
        &=\xi.
    \end{align*}
    Then the conclusion follows by the uniqueness of Jacobi fields.
\end{proof}

\begin{prop}[Gauss Lemma]
    $\langle\exp_{p*}|_v(\xi),\dot\gamma_v(1)\rangle=\langle\xi,v\rangle$.
\end{prop}
\begin{proof}
    Let $J$ be a Jacobi field along $\gamma_v$ with $J(0)=0$, $\dot{J}(0)=\xi$.
    Then by Proposition~\ref{compute exp*}, we have
    \begin{align*}
        \langle\exp_{p*}|_v(\xi),\dot\gamma_v(1)\rangle&=\langle J(1),\dot\gamma_v(1)\rangle.
    \end{align*}
    We differentiate above inner product, obtaining
    \[\frac{\d{}}{\d{t}}\langle J(t),\dot\gamma_v(t)\rangle=\langle\dot{J}(t),\dot\gamma_v(t),\rangle\]
    and
    \begin{align*}
        \frac{\d{}}{\d{t}}\langle\dot{J}(t),\dot\gamma_v(t)\rangle&=\langle\ddot{J}(t),\dot\gamma_v(t)\rangle\\
        &=-\langle R(\dot\gamma_v,J)\dot\gamma_v,\dot\gamma_v\rangle\\
        &=0.
    \end{align*}
    Hence $\langle\dot{J}(t),\dot\gamma_v(t)\rangle$ is constant.
    Clearly $\langle\dot{J}(0),\dot\gamma_v(0)\rangle=\langle\xi,v\rangle$, then we have
    \begin{align*}
        \langle J(1),\dot\gamma_v(1)\rangle-\langle J(0),\dot\gamma_v(0)\rangle&=\int_0^1\langle\xi,v\rangle\d{t}\\
        &=\langle\xi,v\rangle.
    \end{align*}
    Notice that $\langle J(0),\dot\gamma_v(0)\rangle=0$ and we obtain the result.
\end{proof}

Finally we consider the case where $\exp_{p*}$ degenerates.
\begin{defn}
    Let $\gamma$ be a geodesic starting at $p$, if $\exp_{p*}|_{\gamma(t_0)}$ is degenerate at $\gamma(t_0)$, then $\gamma(t_0)$ is called a \emph{conjugate point} of $p$.
\end{defn}

A simple criterion for conjugate points is
\begin{prop}
    $\exp_{p*}$ degenerates at $\gamma(t_0)$ if and only if there is a nontrivial Jacobi field $J$ with $J(0)=J(t_0)=0$.
\end{prop}
\begin{proof}
    $\exp_{p*}|_{\gamma(t_0)}(\xi)=0$ if and only if Jacobi field $J(t)=\frac{\partial{}}{\partial{s}}\exp_p(t(v+s\xi))$ satisfies $J(0)=J(t_0)=0$.
\end{proof}

\begin{rem}
    This proposition shows conjugate is symmetric.
\end{rem}

\section{Index Form}

Using the first variation of energy, we conclude that a length-minimizing curve must be a geodesic.
However, the converse is generally not true.
For instance, a geodesic on sphere starting from the north pole and crossing the south pole is not length-minimizing.
We need to find a sufficient condition for a geodesic to be length-minimizing.
Calculus hints to us that we should differentiate energy twice.
This lead to the second variation formula.

\begin{prop}[Second variation formula]
    Let $\gamma:[0,a]\times(-\varepsilon,\varepsilon)$ be a variation, and $\gamma_0(t)=\gamma(t,0)$ is a geodesic, then
    \[E''(0)=\int_0^a|\dot{V}(t)|^2-\langle R(\dot\gamma_0(t),V(t))\dot\gamma_0(t),V(t)\rangle\d{t}+\boxed{\langle\nabla_{V(t)}V(t),\dot\gamma_0(t)\rangle|^a_0}.\]
    The boxed term is called \emph{boundary term}, and it vanishes when the variation is proper.
\end{prop}
\begin{proof}
    We take the expression of $E'(s)$ in the first variation formula and differentiate, we have
    \begin{align*}
        E''(s)&=\int_0^a\frac{\partial{}}{\partial{s}}\left\langle\frac{\partial\gamma}{\partial{t}},\nabla_{\frac{\partial\gamma}{\partial{s}}}\frac{\partial\gamma}{\partial{t}}\right\rangle\d{t}\\
        &=\int_0^a\left\langle\nabla_{\frac{\partial\gamma}{\partial{s}}}\frac{\partial\gamma}{\partial{t}},\nabla_{\frac{\partial\gamma}{\partial{s}}}\frac{\partial\gamma}{\partial{t}}\right\rangle+\left\langle\frac{\partial\gamma}{\partial{t}},\nabla_{\frac{\partial\gamma}{\partial{s}}}\nabla_{\frac{\partial\gamma}{\partial{s}}}\frac{\partial\gamma}{\partial{t}}\right\rangle\d{t},
    \end{align*}
    where
    \[\left\langle\nabla_{\frac{\partial\gamma}{\partial{s}}}\frac{\partial\gamma}{\partial{t}},\nabla_{\frac{\partial\gamma}{\partial{s}}}\frac{\partial\gamma}{\partial{t}}\right\rangle=\left\langle\nabla_{\frac{\partial\gamma}{\partial{t}}}\frac{\partial\gamma}{\partial{s}},\nabla_{\frac{\partial\gamma}{\partial{t}}}\frac{\partial\gamma}{\partial{s}}\right\rangle=|\dot{V}(t)|^2\]
    when $s=0$, and
    \[\nabla_{\frac{\partial\gamma}{\partial{s}}}\nabla_{\frac{\partial\gamma}{\partial{s}}}\frac{\partial\gamma}{\partial{t}}=\nabla_{\frac{\partial\gamma}{\partial{s}}}\nabla_{\frac{\partial\gamma}{\partial{t}}}\frac{\partial\gamma}{\partial{s}}=\nabla_{\frac{\partial\gamma}{\partial{t}}}\nabla_{\frac{\partial\gamma}{\partial{t}}}\frac{\partial\gamma}{\partial{s}}-R\left(\frac{\partial\gamma}{\partial{s}},\frac{\partial\gamma}{\partial{t}}\right)\frac{\partial\gamma}{\partial{s}}.\]
    Thus when $s=0$, we have
    \[E''(0)=\int_0^a|\dot{V}(t)|^2-\langle R(\dot\gamma_0,V)\dot\gamma_0,V\rangle+\frac{\partial}{\partial{t}}\langle\dot\gamma_0,\nabla_VV\rangle-\langle\nabla_{\dot\gamma_0}\dot\gamma_0,\nabla_VV\rangle,\]
    the last term is $0$ since $\gamma_0$ is a geodesic, and we obtain the formula.
\end{proof}

Second variation formula gives rise to a symmetric bilinear form on the vector space of vector fields along a geodesic $\gamma$, which we will call it index form, to deduce whether a geodesic is locally length minimizing.

\begin{defn}
    Let $\gamma:[0,a]\to M$ be a geodesic.
    The \emph{index form} of $\gamma$ is a bilinear form on $\Gamma(TM,\gamma)$ defined by
    \[I(X,Y)=\int_0^a\langle\dot{X},\dot{Y}\rangle-\langle R(\dot\gamma,X)\dot\gamma,Y\rangle\d{t}.\]
\end{defn}

Clearly index form is symmetric.

\begin{lem}\label{Jacobi field as null space}
    Let $U\in\Gamma_0(TM,\gamma)=\{Y\in\Gamma(TM,\gamma):\ Y(0)=Y(a)=0\}$, then $U$ is a Jacobi field if and only if $I(U,Y)=0$ for any $Y\in\Gamma_0(TM,\gamma)$.
\end{lem}
\begin{proof}
    First we observe that by integrate by parts, we have
    \[I(U,Y)=-\int_0^a\langle\ddot{U}+R(\dot\gamma,U)\dot\gamma,Y\rangle.\]
    If $U$ is a Jacobi field, then $\ddot{U}+R(\dot\gamma,U)\dot\gamma=0$, which implies $I(U,Y)=0$.
    Conversely, if $I(U,Y)=0$ for any $Y\in\Gamma_0(TM,\gamma)$, choose a smooth function $f$ with $f(0)=f(a)=0$ and positive elsewhere, let $Y=f(t)(\ddot{U}+R(\dot\gamma,U)\dot\gamma)$, then
    \[0=-f\int_0^a|\ddot{U}+R(\dot\gamma,U)\dot\gamma|^2\d{t},\]
    this implies $\ddot{U}+R(\dot\gamma,U)\dot\gamma=0$, that is, $U$ is a Jacobi field.
\end{proof}

We observe if a Jacobi field $J$ does not vanish at the starting or ending point of the geodesic, the above integration by parts is changed into
\begin{align*}
    I(Y,J)&=\langle Y,\dot{J}\rangle|_0^a-\int_0^a\langle\ddot{J}+R(\dot\gamma,J)\dot\gamma,Y\rangle\d{t}\\
    &=\langle Y,\dot{J}\rangle|_0^a.
\end{align*}

The next proposition shows the positive definiteness of $I$ is related to conjugate points.
\begin{prop}\label{prejacobi}
    Let $\gamma:[0,a]\to M$ be a geodesic, $I$ be its index form.
    \begin{enumerate}[(1)]
        \item If $\gamma$ contains no conjugate points of $\gamma(0)$, then $I$ is positive definite on $\Gamma_0(TM,\gamma)$.
        \item If $\gamma(a)$ is the only conjugate point of $\gamma(0)$, then $I$ is positive semidefinite but not positive definite.
        \item If there is a $t_0<a$ such that $\gamma(t_0)$ is conjugate to $\gamma(0)$, then there is a $U\in\Gamma_0(TM,\gamma)$ such that $I(U,U)<0$. 
    \end{enumerate}
\end{prop}
\begin{proof}[Proof of Proposition~\ref{prejacobi}~(1)]
    Let $\gamma(0)=p$, $\tilde\gamma$ is the radial line in $T_pM$ defined by $\tilde\gamma(t)=t\dot\gamma(0)$.
    Since $\gamma$ contains no conjugate points of $\gamma(0)$, $\exp_p$ is nondegenerate on the whole $\tilde\gamma$.
    Hence there is a neighborhood $U$ of $\tilde\gamma([0,a])$ such that $\exp_p:U\to M$ is an immersion.
    By Proposition~\ref{geodesic locally length-min}~and the subsequent remark, $\gamma$ is length-minimizing.
    Hence by Proposition~\ref{length-min to energy-min}, $\gamma$ also minimizes energy.
    Let $\gamma(t,s):[0,a]\times(-\varepsilon,\varepsilon)\to M$ be any proper variation, by taking $\varepsilon$ small enough, we can assume the variation is contained in $U$.
    Then we have
    \[E''(0)=\lim_{s\to 0}\frac{E(-s)+E(s)-2E(0)}{s^2}\geq 0.\]
    Since $E''(0)(V)=I(V,V)$ for any variation vector field, we have $I(V,V)\geq 0$ for any $V\in\Gamma_0(TM,\gamma)$.
    Now we must show $I(V,V)=0$ implies $V=0$.
    Let $I(V,V)=0$, $X\in\Gamma_0(TM,\gamma)$ and $\delta>0$, we have
    \[0\leq I(V+\delta X,V+\delta X)=I(V,V)+2\delta I(V,X)+\delta^2I(X,X),\]
    this implies
    \[2I(V,X)+\delta I(X,X)\geq 0.\]
    Let $\delta\to 0$, we obtain $I(V,X)\geq 0$ for any $X\in\Gamma_0(TM,\gamma)$.
    Similarly, consider $I(V-\delta X,V-\delta X)$, we have $I(V,X)\leq 0$ for any $X\in\Gamma_0(TM,\gamma)$.
    This means $I(V,X)=0$ for any $X\in\Gamma_0(TM,\gamma)$.
    Now by Lemma~\ref{Jacobi field as null space}, $V$ is a Jacobi field.
    But $\gamma$ has no conjugate points of $\gamma(0)$, $V$ must identically equal to $0$.
    This proves $I$ is positive definite.
\end{proof}

Before proving the rest of Theorem~\ref{prejacobi}, we need the following \emph{Index Lemma}.

\begin{prop}[Index Lemma]
    Assume $\gamma:[0,a]\to M$ is a geodesic without conjugate points.
    Let $U\in\Gamma(TM,\gamma)$ with $U(0)=0$, $J$ be a Jacobi field such that $J(0)=0$, $J(a)=U(a)$.
    Then $I(J,J)\leq I(U,U)$, the equality holds if and only if $U=J$.
\end{prop}
\begin{proof}
    Since $U-J\in\Gamma_0(TM,\gamma)$, by Theorem~\ref{prejacobi}~(1), we have $I(U-J,U-J)\geq 0$ with equality holds iff $U=J$.
    Then we have
    \[0\leq I(U-J,U-J)=I(U,U)-2I(U,J)+I(J,J).\]
    But $I(U,J)=\langle U,\dot{J}\rangle|^a_0=\langle J,\dot{J}\rangle|^a_0=I(J,J)$, thus
    \[I(U,U)\geq I(J,J).\qedhere\]
\end{proof}

\begin{proof}[Proof of the rest of Proposition~\ref{prejacobi}]
    (2) For any $0<t_0<a$, let $I_{[0,t_0]}$ denote the index form of $\gamma|_{[0,t_0]}$, i.e.,
    \[I_{[0,t_0]}(X,Y)=\int_{0}^{t_0}\langle \dot{X},\dot{Y}\rangle+\langle R(\dot\gamma,X)\dot\gamma,Y\rangle\d{t}.\]
    Then for $X\in\Gamma_0(TM,\gamma|_{[0,t_0]})$, by (1) we have $I_{[0,t_0]}(X,X)\geq 0$.
    We now construct $\tau_{t_0}(U)$ for any $U\in\Gamma_0(TM,\gamma)$.
    Let $\{e_i(t)\}$ be a parallel frame along $\gamma$, $U=f^i(t)e_i(t)$.
    Then we define
    \[\tau_{t_0}(U)(t)=f^i\left(\frac{a}{t_0}t\right)e_i\left(\frac{a}{t_0}t\right)\in\Gamma_0(TM,\gamma|_{[0,t_0]}).\]
    Thus we have
    \[I_{[0,t_0]}(\tau_{t_0}(U),\tau_{t_0}(U))\geq 0,\]
    let $t_0\to a$ we obtain the conclusion.
    Moreover, since $\gamma(a)$ is conjugate to $\gamma(0)$, there exists a nontrivial Jacobi field $J$ with $J(0)=J(a)=0$, and $I(J,J)=\langle J,\dot{J}\rangle|^a_0=0$.
    This shows $I$ is positive semidefinite but not positive definite.\\
    (3) Since $\gamma(0)$ is conjugate to $\gamma(t_0)$, there exists a Jacobi field $\tilde{J}_1$ along $\gamma|_{[0,t_0]}$ such that $\tilde{J}_1(0)=\tilde{J}_1(t_0)=0$.
    Choose a smooth $m_1:[0,a]\to\mathbb{R}$ with $\supp{m_1}=[0,t_0]$, and set $J_1=m_1\tilde{J}_1$.
    Let
    \[V(t)=\begin{cases}
        J_1(t), & 0\leq t\leq t_0,\\
        0, & t_0\leq t\leq a,
    \end{cases}\]
    then $V\in\Gamma(TM,\gamma)$ is smooth.
    Let $\delta>0$ so small that $\gamma_{[t_0-\delta,t_0+\delta]}$ is contained in a normal neighborhood of $\gamma(t_0)$.
    Then there exists a Jacobi field $\tilde{J}_2$ along $\gamma|_{[t_0-\delta,t_0+\delta]}$ with $\tilde{J}_2(t_0-\delta)=J_1(t_0-\delta)$, $\tilde{J}_2(t_0+\delta)=0$.
    Let $m_2:[t_0-\delta,t_0+\delta]$ satisfy
    \[\dot{J}_1=\nabla_{\dot\gamma}m_2\tilde{J}_2=\frac{\d{m_2}}{\d{t}}\tilde{J}_2+m_2\dot{\tilde{J}}_2\]
    and
    \[R(\dot\gamma(t_0-\delta),J_1(t_0-\delta))\dot\gamma(t_0-\delta)=m_2(t_0-\delta)R(\dot\gamma(t_0-\delta),\tilde{J}_2(t_0-\delta))\dot\gamma(t_0-\delta),\]
    then by ODE theory, $m_2$ is the solution of a linear ODE of order $1$, hence it exists and is smooth.
    Now set $J_2=m_2\tilde{J_2}$, and
    \[U(t)=\begin{cases}
        J_1(t), & 0\leq t\leq t_0-\delta,\\
        J_2(t), & t_0-\delta\leq t\leq t_0+\delta,\\
        0, & t_0+\delta\leq t\leq a,
    \end{cases}\]
    then up to a third mollification, $U(t)\in\Gamma(TM,\gamma)$.
    Therefore by Index Lemma, we have
    \begin{align*}
        I(U,U)&=I_{[0,t_0-\delta]}(J_1,J_1)+I_{[t_0-\delta,t_0+\delta]}(J_2,J_2)+0\\
        &<I_{[0,t_0-\delta]}(V,V)+I_{[t_0-\delta,t_0+\delta]}(V,V)\\
        &=I_{[0,t_0]}(V,V)+I_{[t_0,a]}(V,V)\\
        &=\langle J_1,\dot{J}_1\rangle|_0^{t_0}+0\\
        &=0.\qedhere
    \end{align*}
\end{proof}

Translating Proposition~\ref{prejacobi}~to the language of geometry, we have the following theorem.
\begin{thm}[Jacobi]
    Let $\gamma:[0,a]\to M$ be a geodesic, then
    \begin{enumerate}[(1)]
        \item If $\gamma$ has no conjugate points of $\gamma(0)$, then $\gamma$ is length-minimizing under any small proper variation.
        \item If there is a $t_0\in(0,a)$ such that $\gamma(t_0)$ is conjugate to $\gamma(0)$, then there is a proper variation $\gamma_s$ such that $L[\gamma_s]<L[\gamma]$ for any $s$.
    \end{enumerate}
\end{thm}

\section{Bochner Formula}
We introduce a useful calculation tool.
We will later apply it to distance function.

\begin{thm}[Bochner formula]
    Let $M$ be a Riemannian manifold, $f\in C^\infty(M)$.
    Then we have
    \[\Delta\frac{1}{2}|\nabla f|^2=|\nabla^2f|^2+\langle\nabla\Delta f,\nabla f\rangle+\langle\Ric(\nabla f),\nabla f\rangle.\]
\end{thm}

We first need a lemma.
\begin{lem}
    Under geodesic coordinate around $p$, the Ricci identity (Proposition~\ref{Ricci identity}) of $1$-forms is equivalent to
    \[f_{;kij}-f_{;kji}=-f_{;l}\R{l}{i}{j}{k},\]
    where $f_{;i}$ means $\partial_if$.
\end{lem}
\begin{proof}
    On the one hand, since all Christoffel symbols vanish at $p$, we have
    \begin{align*}
        \nabla_{\partial_i}\nabla_{\partial_j}\d{f}&=\nabla_{\partial_i}\nabla_{\partial_j}f_{;k}\d{x^k}\\
        &=\nabla_{\partial_i}(\partial_jf_{;k}\d{x^k}+f_{;k}\nabla_{\partial_j}\d{x^k})\\
        &=\nabla_{\partial_i}(\partial_jf_{;k}\d{x^k}-f_{;k}\Gamma^k_{jl}\d{x^l})\\
        &=\partial_i\partial_jf_{;k}\d{x^k}+\partial_jf_{;k}\nabla_{\partial_i}\d{x^k}\\
        &=f_{;kji}-\partial_jf_{;k}\Gamma^k_{il}\d{x^l}\\
        &=f_{;kji}\d{x^k},
    \end{align*}
    then we have
    \begin{align*}
        (R(\partial_i,\partial_j)\d{f})(\partial_k)&=(\nabla_{\partial_j}\nabla_{\partial_i}-\nabla_{\partial_i}\nabla_{\partial_j}+\nabla_{[\partial_i,\partial_j]})\d{f}(\partial_k)\\
        &=f_{;kij}-f_{;kji}.
    \end{align*}
    On the other hand, we have
    \begin{align*}
        (R(\partial_i,\partial_j)\d{f})(\partial_k)&=-\d{f}(R(\partial_i,\partial_j)\partial_k)\\
        &=-\d{f}(\R{l}{i}{j}{k}\partial_l)\\
        &=-f_{;m}\d{x^m}(\R{l}{i}{j}{k}\partial_l)\\
        &=-f_{;l}\R{l}{i}{j}{k},
    \end{align*}
    hence the equality holds.
\end{proof}

\begin{proof}[Proof of Bochner formula]
    Under geodesic normal coordinate we have the following calculation:
    \begin{align*}
        \Delta\frac{1}{2}|\nabla f|^2&=\sum_{i,j}\left(\frac{1}{2}f_{;j}^2\right)_{;ii}\\
        &=\sum_{i,j}(f_{;j}f_{;ji})_{;i}\\
        &=\sum_{i,j}(f_{;ji}^2+f_{;j}f_{;jii})\\
        &=\sum_{i,j}(f_{;ij}^2+f_{;j}f_{;iji})\\
        &=\sum_{i,j}f_{;ij}^2+\sum_{i,j}f_{;j}(f_{;iij}-f_{;k}\R{k}{j}{i}{i})\\
        &=\sum_{i,j}f_{;ij}^2+\sum_{i,j}f_{;j}f_{;iij}+\sum_{i,j}f_{;j}f_{;k}\R{k}{i}{j}{i}\\
        &=\sum_{i,j}f_{;ij}^2+\sum_{i,j}f_{;j}f_{;iij}+\sum_{j}f_{;j}f_{;k}\Ricij{k}{j}\\
        &=|\nabla^2f|^2+\langle\nabla f,\nabla\Delta f\rangle+\langle\Ric(\nabla f),\nabla f\rangle.\qedhere
    \end{align*}
\end{proof}

\backmatter
\bibliographystyle{plain}
\bibliography{biblio}

\end{document}