\chapter{Submanifolds}

\section{Second Fundamental Form}

We first set up our stage.
Let $f:\Sigma^k\to(M^n,g)$ be an immersion, equip $\Sigma$ with pullback metric $f^*g$ (still denoted by $g$ for simplicity).
There is a decomposition $\nabla_XY=(\nabla_XY)^\top+(\nabla_XY)^\perp$ for $X,Y\in\Gamma(T\Sigma)$, and simple observation shows
\begin{prop}
    Let $\nabla^\Sigma$ be the Levi--Civita connection on $(\Sigma,g)$, then $\nabla^\Sigma$ is given by
    \[\nabla^\Sigma_XY=(\nabla_XY)^\top\]
    for $X,Y\in\Gamma(T\Sigma)$.
\end{prop}

Now we need the concept of normal bundle.
\begin{defn}
    Let $\Sigma$ be a submanifold of $M$, then for any $p\in M$ we have the decomposition
    \[T_pM=T_p\Sigma\oplus N_p\Sigma,\]
    where $N_p\Sigma$ is the orthogonal complement of $T_p\Sigma$.
    Then define the \emph{normal bundle} of $\Sigma$ to be
    \[N\Sigma=\bigsqcup_{p\in M}N_p\Sigma.\]
    It is similar to tangent bundle to make $N\Sigma$ into a vector bundle.
\end{defn}

Thus we can define the second fundamental form of a submanifold.
\begin{defn}
    Let $\Sigma$ be a submanifold of $M$, then we define the \emph{second fundamental form} of $\Sigma$ to be
    \begin{align*}
        \II:\Gamma(T\Sigma)\times\Gamma(T\Sigma)&\to\Gamma(N\Sigma)\\
        (X,Y)&\mapsto(\nabla_XY)^\perp.
    \end{align*}
\end{defn}
\begin{lem}
    The second fundamental form is symmetric.
\end{lem}
\begin{proof}
    We have
    \[\II(X,Y)-\II(Y,X)=(\nabla_XY)^\perp-(\nabla_YX)^\perp=([X,Y])^\perp=0.\qedhere\]
\end{proof}

Moreover, we can define the shape operator.
\begin{defn}
    For a fixed $\xi\in\Gamma(N\Sigma)$, we define
    \begin{align*}
        S_\xi:\Gamma(T\Sigma)&\to\Gamma(T\Sigma)\\
        X&\mapsto\nabla_X\xi-(\nabla_X\xi)^\perp.
    \end{align*}
\end{defn}

We have the following Weingarten formula.
\begin{prop}
    For $\xi\in\Gamma(N\Sigma)$ and $X,Y\in\Gamma(T\Sigma)$, we have
    \[\langle S_\xi(X),Y\rangle=\langle\II(X,Y),-\xi\rangle.\]
    In particular, shape operator is self-adjoint.
\end{prop}
\begin{proof}
    We have the calculation
    \begin{align*}
        \langle S_\xi(X),Y\rangle&=\langle\nabla_X\xi-(\nabla_X\xi)^\perp,Y\rangle\\
        &=\langle\nabla_X\xi,Y\rangle-\langle(\nabla_X\xi)^\perp,Y\rangle\\
        &=X\langle\xi,Y\rangle-\langle\xi,\nabla_XY\rangle\\
        &=\langle-\xi,\nabla_XY\rangle\\
        &=\langle-\xi,(\nabla_XY)^\perp\rangle\\
        &=\langle\II(X,Y),-\xi\rangle.\qedhere
    \end{align*}
\end{proof}

For a two-sided hypersurface $\Sigma^{n-1}\subset M^n$, choose a unit normal vector field $N$, we can introduce the \emph{second fundamental form tensor} $h$, which is defined by $h(X,Y):=\langle\II(X,Y),N\rangle$.
In other words, $\II(X,Y)=h(X,Y)N$, and the Weingarten formula is $h(X,Y)=-\langle S(X),Y\rangle$.

\section{Gauss--Codazzi Equations}

In this section, our goal is to prove the following Gauss--Codazzi equations on the relation of curvature between a hypersurface and ambient manifold.
After this, we give a brief discussion on principal curvatures.

In the following contents, we assume $\Sigma^{n-1}\subset M^{n-1}$, $\Sigma$ is two-sided.
We will use $\D_XY$ to denote the Levi--Civita connection of $M$, and $\nabla_XY$ for $\Sigma$.
$N$ will denote the unit normal vector field of $\Sigma$.

\begin{thm}
    Under above settings, for $X,Y,Z,W\in\Gamma(T\Sigma)$ and $N\in\Gamma(N\Sigma)$, we have
    \begin{itemize}
        \item Gauss equation:
        \begin{align*}
            R^\Sigma(X,Y,Z,W)=&R^M(X,Y,Z,W)+h(X,Z)h(Y,W)-h(X,W)h(Y,Z);
        \end{align*}
        \item Codazzi equation:
        \[R^M(X,Y,Z,N)=(\nabla_Yh)(X,Z)-(\nabla_Xh)(Y,Z).\]
    \end{itemize}
\end{thm}
\begin{proof}
    We first calculate $R^\Sigma(X,Y)Z$.
    We have
    \begin{align*}
        R^\Sigma(X,Y)Z=&\nabla_Y\nabla_XZ-\nabla_X\nabla_YZ+\nabla_{[X,Y]}Z\\
        =&\D_Y\nabla_XZ-h(\nabla_XZ,Y)N-\D_X\nabla_YZ+h(\nabla_YZ,X)N\\
        &+D_{[X,Y]}Z-h([X,Y],Z)N\\
        =&\D_Y\D_XZ-\D_Yh(X,Z)N-h(\nabla_XZ,Y)N\\
        &-\D_X\D_YZ+\D_Xh(Y,Z)N+h(\nabla_YZ,X)N\\
        &+\D_{[X,Y]}Z-h(\nabla_XY-\nabla_YX,Z)N\\
        =&R^M(X,Y)Z-\D_Yh(X,Z)N+h(\nabla_YX,Z)+h(X,\nabla_YZ)N\\
        &+\D_Xh(Y,Z)N-h(\nabla_XY,Z)N-h(Y,\nabla_XZ)N\\
        =&R^M(X,Y)Z\\
        &-Yh(X,Z)N-h(X,Z)\D_YN+h(\nabla_YX,Z)+h(X,\nabla_YZ)N\\
        &+Xh(Y,Z)N+h(Y,Z)\D_XN-h(\nabla_XY,Z)N-h(Y,\nabla_XZ)N.
    \end{align*}
    First we consider $\langle R^\Sigma(X,Y)Z,W\rangle$, by Weingarten formula we have
    \begin{align*}
        \langle R^\Sigma(X,Y)Z,W\rangle&=\langle R^M(X,Y)Z,W\rangle+\langle h(Y,Z)\D_XN,W\rangle-\langle h(X,Z)\D_YN,W\rangle\\
        &=\langle R^M(X,Y)Z,W\rangle+h(X,Z)h(Y,W)-h(X,W)h(Y,Z).
    \end{align*}
    Next we consider $\langle R^\Sigma(X,Y)Z,N\rangle$, notice that $\langle D_XN,N\rangle=\frac{1}{2}X\langle N,N\rangle=0$, we have
    \begin{align*}
        0=&\langle R^M(X,Y)Z,N\rangle+Xh(Y,Z)-h(\nabla_XY,Z)-h(Y,\nabla_XZ)\\
        &-Yh(X,Z)+h(\nabla_YX,Z)+h(X,\nabla_YZ)\\
        =&\langle R^M(X,Y)Z,N\rangle+(\nabla_Xh)(Y,Z)-(\nabla_Yh)(X,Z),
    \end{align*}
    which is equivalent to Codazzi equation.
\end{proof}

We now turn to principal curvatures.

\begin{defn}
    Let $p\in\Sigma^{n-1}\subset M^n$, the \emph{principal curvatures} at $p$ are the eigenvalues of shape operator $S$.
    Usually we denote $\kappa_i$, $i=1,\cdots,n-1$ for principal curvatures.
\end{defn}

Principal curvatures are extrinsic geometry quantities, but on $2$-dimensional surfaces, their product happens to be intrinsic.
This is the \emph{Theorema Egregium}, a glorious achievement of classical differential geometry.
At the era of Gauss, who discovered the theorem, Theorema Egregium needs tedious calculation to be proved.
But using modern language, Theorema Egregium is almost self-evident.
We now state and prove this theorem.

\begin{thm}[Theorema Egregium]
    On a surface $\Sigma^2\subset\mathbb{R}^3$, we define \emph{Gaussian curvature} $\kappa=\kappa_1\kappa_2$, the product of principal curvatures.
    The Gaussian curvature is equal to the sectional curvature on a point, hence Gaussian curvature is intrinsic, i.e.\ only depends on metric.
\end{thm}
\begin{proof}
    Since $S^i_j=-g^{ik}h_{kj}$, we have
    \[\kappa=\kappa_1\kappa_2=\det(S^i_j)=\frac{\det(h_{ij})}{\det(g_{ij})}.\]
    However, by Gauss equation, we have
    \begin{align*}
        \Sect_p(T_p\Sigma)&=\frac{\langle R(\partial_i,\partial_j)\partial_i,\partial_j\rangle}{g_{ii}g_{jj}-g_{ij}^2}\\
        &=\frac{1}{\det(g_{ij})}(0+h_{ii}h_{jj}-h_{ij}^2)\\
        &=\kappa.\qedhere
    \end{align*}
\end{proof}

Finally we discuss totally umbilical hypersurfaces.
\begin{defn}
    A point $p\in\Sigma^{n-1}\subset M^n$ is called \emph{umbilical} if $h_p=cg_p$ for some $c\in\mathbb{R}$.
    $\Sigma^{n-1}$ is called \emph{totally umbilical} if $h=fg$ for some $f\in C^\infty(M)$.
\end{defn}

\begin{lem}\label{totally umbilical}
    A point is umbilical if and only if all the principal curvatures at this point are equal.
\end{lem}
\begin{proof}
    This follows by $S^i_j=-g^{ik}h_{kj}=-fg^{ik}g_{kj}=-f\delta^i_j$.
\end{proof}

\begin{prop}
    If $\Sigma$ is a totally umbilical hypersurface in $(\mathbb{R}^n,\delta)$, i.e.\ $h=f\delta$, then $f$ must be constant.
\end{prop}
\begin{proof}
    By Codazzi equation, at $p\in\Sigma$, for $u,w,v\in T_p\Sigma$, we have
    \begin{equation}
        0=\langle R(u,v)w,n\rangle=(\nabla h)(u,w,v)-(\nabla h)(v,w,u).\label{codazzi umbilical}
    \end{equation}
    Since $h=f\delta$, we have
    \[\nabla h=\nabla (f\delta)=\delta\otimes\d{f}+f\nabla\delta=\delta\otimes\d{f}.\]
    Hence equation~\ref{codazzi umbilical}~is equivalent to
    \[0=\langle u,w\rangle\d{f}_p(v)-\langle v,w\rangle\d{f}|_p(u).\]
    Choose $v$ arbitrarily, $w\neq 0$ satisfying $\langle v,w\rangle=0$ and $u=w$, we have $\d{f}|_p(v)=0$.
    This shows $\d{f}|_p=0$.
    Since $p$ is arbitrary, $\d{f}=0$ and this implies $f\equiv\text{const}$. 
\end{proof}

\begin{rem}
    Some authors mistakenly think this proposition holds for general ambient manifold.
    We have a counterexample as follows.
    Let's consider the warped product $\mathbb{R}\times\mathbb{R}^2$ with metric $g=(\d{x^1})^2+F^2(x^1)((\d{x^2})^2+(\d{x^3})^2)$, where $F$ is a nonconstant function of $x^1$.
    If $\{\partial_i\}$ is coordinate vector field, then $N=\partial_1$ is orthonormal to $\Sigma=\mathbb{R}^2$.
    We have
    \[\Gamma^1_{13}=\frac{1}{2}g^{11}(\partial_3g_{11}+\partial_1g_{13}-\partial_1g_{13})=\frac{\partial_3F}{F}=\Gamma^2_{23},\]
    and
    \[\Gamma^2_{13}=\frac{1}{2}g^{22}(\partial_3g_{12}+\partial_1g_{23}-\partial_2g_{13})=0.\]
    Therefore we have
    \[h(\partial_1,\partial_1)=-\langle\nabla_{\partial_1}\partial_3,\partial_1\rangle=-F^2\Gamma^1_{13}=-F\partial_3F=h(\partial_2,\partial_2),\]
    and
    \[h(\partial_1,\partial_2)=-\langle\nabla_{\partial_1}\partial_3,\partial_2\rangle=-F^2\Gamma^2_{13}=0.\]
    Hence we have
    \[h=-\frac{\partial_3F}{F}g=:fg,\]
    $f$ is not constant as $F$ is not constant.
\end{rem}

If we assume a priori the classification theorem of constant sectional curvature spaces, we can have the following corollary.
\begin{cor}
    A complete totally umbilical hypersurface $\Sigma$ in $(\mathbb{R}^n,\delta)$ is either a hyperplane or a sphere.
\end{cor}
\begin{proof}
    Let $p\in\Sigma$, $u,v\in T_p\Sigma$, then by Gauss equation, we have
    \begin{align*}
        \Sect_p(\Span\{u,v\})&=0+\frac{h(u,u)h(v,v)-h^2(u,v)}{\delta(u,u)\delta(v,v)-\delta^2(u,v)}\\
        &=f^2\frac{\delta(u,u)\delta(v,v)-\delta^2(u,v)}{\delta(u,u)\delta(v,v)-\delta^2(u,v)}\\
        &=f^2\\
        &\equiv\text{const}\geq 0.
    \end{align*}
    When the constant is $0$, $\Sigma$ is a hyperplane; when the constant is positive, $\Sigma$ is a sphere.
\end{proof}

\section{Minimal Surfaces}

We now give a brief introduction to minimal surfaces.
We will use the word ``area'' to denote the volume of a submanifold.

\begin{defn}
    Let $\Sigma^k\subset M^n$ be a submanifold (maybe with boundary), the \emph{area functional} is defined as
    \[A[\Sigma]=\int_{\Sigma}\d{A_\Sigma},\]
    where $A_\Sigma$ is the volume element on $\Sigma$.
\end{defn}

The guiding problem of this section is
\begin{pro}[Generalized Plateau problem]
    Given a fixed boundary submanifold $B^{k-1}\subset M^n$ ($k<n$), find the area-minimi\-zer in all submanifolds $\Sigma^k$ with $\partial\Sigma=B$.
\end{pro}

We use variation to deduce a necessary condition for a submanifold to be area-minimizing.
To describe the first variation formula of area, we introduce the notion of mean curvature.

\begin{defn}
    Let $\Sigma^k\subset M$, define the \emph{mean curvature vector} at a point $p\in\Sigma$ to be
    \[\mathbf{H}_p=\sum_{i=1}^k\II_p(e_i,e_i),\]
    where $\{e_i\}$ is an orthonormal basis of $T_p\Sigma$.
    When $k=n-1$, i.e.\ $\Sigma$ is a hypersurface, we define \emph{mean curvature} $H=\tr h$, the trace of second fundamental form tensor.
\end{defn}

\begin{prop}[First variation of area]
    Let $M^n$ be a Riemannian manifold, $\Sigma^k$ be a manifold with boundary with $k<n$, $f:\Sigma\times(-\varepsilon,\varepsilon)\to M$ be a family of immersions, denote $\Sigma_t=f(\Sigma,t)$.
    Suppose $f(p,t)=p$ for $p\in\partial\Sigma$ and $t\in(-\varepsilon,\varepsilon)$, the variation vector field $X:=\left.\frac{\partial}{\partial{t}}\right|_{t=0}f\in\Gamma(TM,\Sigma_0)$.
    Then we have
    \[\left.\frac{\d{}}{\d{t}}\right|_{t=0}A[\Sigma_t]=-\int_{\Sigma_0}\langle\mathbf{H},X\rangle\d{A_{\Sigma_0}}.\]
\end{prop}
\begin{proof}
    Denote $g_t=f^*_tg$ and $A_t=A_{\Sigma_t}$.
    Let
    \[\mathscr{J}(p,t)=\frac{\sqrt{\det{g_t}}}{\sqrt{\det{g_0}}},\]
    then we have $A_t(p)=\mathscr{J}(p,t)A_0(p)$.
    Now we can calculate (omitting $p$)
    \begin{align*}
        \left.\frac{\d{}}{\d{t}}\right|_{t=0}\mathscr{J}(t)&=\frac{1}{\sqrt{\det{g_0}}}\cdot\frac{1}{2\sqrt{\det{g_0}}}\cdot\det{g_0}\cdot g_0^{ij}\left.\frac{\d{}}{\d{t}}\right|_{t=0}g_{ij}\\
        &=\frac{1}{2}g_0^{ij}\left(\left\langle\nabla_{\partial_t}\frac{\partial{f}}{\partial{x^i}},\frac{\partial{f}}{\partial{x^j}}\right\rangle+\left\langle\frac{\partial{f}}{\partial{x^i}},\nabla_{\partial_t}\frac{\partial{f}}{\partial{x^j}}\right\rangle\right)\\
        &=g_0^{ij}\left\langle\nabla_{\partial_t}\frac{\partial{f}}{\partial{x^i}},\frac{\partial{f}}{\partial{x^j}}\right\rangle\\
        &=g_0^{ij}\left\langle\nabla_{\partial_i}\frac{\partial{f}}{\partial{t}},\frac{\partial{f}}{\partial{x^j}}\right\rangle\\
        &=g_0^{ij}\left\langle\nabla_{\partial_i}X^\perp,\frac{\partial{f}}{\partial{x^j}}\right\rangle+g_0^{ij}\left\langle\nabla_{\partial_i}X^\top,\frac{\partial{f}}{\partial{x^j}}\right\rangle\\
        &=-\left\langle g_0^{ij}\II\left(\frac{\partial{f}}{\partial{x^i}},\frac{\partial{f}}{\partial{x^j}}\right),X^\perp\right\rangle+\div{X^\top}\\
        &=-\langle\mathbf{H},X^\perp\rangle+\div{X^\top}.
    \end{align*}
    Notice that $-\langle\mathbf{H},X^\perp\rangle=-\langle\mathbf{H},X\rangle$, take integral we have
    \[\left.\frac{\d{}}{\d{t}}\right|_{t=0}A[\Sigma_t]=-\int_{\Sigma_0}\langle\mathbf{H},X\rangle\d{A_{\Sigma_0}}+\int_{\Sigma_0}\div{X^\top}\d{A_{\Sigma_0}}.\]
    Since the variation fixes boundary, we have $X|_{\partial\Sigma_0}=0$, then by divergence theorem, let $N$ denote the outer unit normal vector field of $\partial\Sigma_0$, we have
    \[\int_{\Sigma_0}\div{X^\top}\d{A_{\Sigma_0}}=\int_{\partial\Sigma_0}\langle X^\top,N\rangle\d{A_{\partial\Sigma_0}}=0.\]
    Hence we proved the first variation formula of area.
\end{proof}

Thus we have a necessary condition for area-minimizing submanifold.
\begin{cor}
    If $\Sigma$ is a solution of generalized Plateau problem, then $\mathbf{H}=0$.
\end{cor}

The corollary gives rise to the notion of minimal submanifolds.

\begin{defn}
    Let $\Sigma$ be a submanifold of $M$, if the mean curvature vector field $\mathbf{H}\equiv 0$, then we call $\Sigma$ a \emph{minimal submanifold}.
\end{defn}

\begin{rem}
    We do not demand a minimal submanifold is with boundary or actually minimzes area.
    But if the minimal submanifold is a graph, then it minimzes area.
\end{rem}

\begin{lem}
    Let $u:\Omega\subset\mathbb{R}^{n-1}\to\mathbb{R}$ be a smooth function, then its graph
    \[G[u]:=\{(x^1,\cdots,x^{n-1},u(x^1,\cdots,x^{n-1}))\}\]
    is a minimal surface if and only if it satisfies the following \emph{minimal surface equation (MSE)}:
    \[\div_{\mathbb{R}^{n-1}}\left(\frac{\D u}{\sqrt{1+|\D u|^2}}\right)=0,\]
    where $\D$ is the Euclidean connection, and norm is Euclidean norm.
\end{lem}
\begin{proof}
    Let
    \begin{align*}
        r:\Omega\subset\mathbb{R}^{n-1}&\to\mathbb{R}^n\\
        (x^1,\cdots,x^{n-1})&\mapsto(x^1,\cdots,x^{n-1},u(x^1,\cdots,x^{n-1})),
    \end{align*}
    $r(\Omega)=G[u]$ has a unit normal vector field
    \[N=\frac{(\D u,-1)}{\sqrt{1+|\D u|^2}}.\]
    Then we have
    \[g_{ij}=\delta_{ij}+\frac{\partial u}{\partial{x^i}}\frac{\partial u}{\partial{x^j}},\]
    or $g=I+\D u(\D u)^t$.
    Hence we have formally
    \begin{align*}
        (I+\D u(\D u)^t)^{-1}&=I-\D u(\D u)^t+(\D u(\D u)^t)^2-(\D u(\D u)^t)^3+\cdots\\
        &=I-\D u(\D u)^t(1-|\D u|^2+|\D u|^4+\cdots)\\
        &=I-\frac{1}{1+|\D u|^2}\D u(\D u)^t,
    \end{align*}
    and one can check we indeed obtain the inverse matrix.
    Moreover, we have second fundamental form tensor
    \[h_{ij}=\langle r_{;ij},N\rangle=-\frac{1}{\sqrt{1+|\D u|^2}}\cdot\frac{\partial^2u}{\partial{x^i}\partial{x^j}}.\]
    Thus by taking trace, we have
    \begin{align*}
        H&=g^{ij}h_{ij}\\
        &=-\sum_{i,j}\left(\delta_{ij}-\frac{1}{1+|\D u|^2}\frac{\partial{u}}{\partial{x^i}}\frac{\partial{u}}{\partial{x^j}}\right)\left(\frac{1}{\sqrt{1+|\D u|^2}}\cdot\frac{\partial^2u}{\partial{x^i}\partial{x^j}}\right)\\
        &=-\sum_{i,j}\frac{\partial_i(\partial_iu)\sqrt{1+|\D u|^2}-\partial_ju\partial_iu\partial_i(\partial_ju)(1+|\D u|^2)^{-1/2}}{1+|\D u|^2}\\
        &=-\sum_{i,j}\frac{\partial{}}{\partial{x^i}}\left(\frac{\partial_iu}{\sqrt{1+|\D u|^2}}\right)\\
        &=-\div_{\mathbb{R}^{n-1}}\left(\frac{\D u}{\sqrt{1+|\D u|^2}}\right),
    \end{align*}
    where we used the fact
    \begin{align*}
        \frac{\partial{}}{\partial{x^i}}\sqrt{1+|\D u|^2}&=\frac{\partial_i(1+\sum_j(\partial_ju)^2)}{2\sqrt{1+|\D u|^2}}\\
        &=\sum_j\frac{\partial_i(\partial_j u)\partial_ju}{\sqrt{1+|\D u|^2}}.
    \end{align*}
    Hence we proved MSE.
\end{proof}

\begin{rem}
    When $n=3$, MSE was found by Lagrange in 1762.
\end{rem}

\begin{prop}
    A minimal graph in $\mathbb{R}^n$ is area-minimizing.
\end{prop}
\begin{proof}
    We need to show if $\Sigma_u$ is a minimal graph, $\Sigma$ is a surface such that $\partial\Sigma_u=\partial\Sigma$, then $A[\Sigma_u]\leq A[\Sigma]$.
    Define a vector field $X\in\Gamma(T\mathbb{R}^n)$ such that
    \[X=\frac{(-\D u,1)}{\sqrt{1+|\D u|^2}}.\]
    Since $X$ does not depend on $x^n$, by MSE we have
    \[\div_{\mathbb{R}^n}X=\div_{\mathbb{R}^{n-1}}\left(\frac{-\D u}{\sqrt{1+|\D u|^2}}\right)=0.\]
    Let $\Omega$ be the domain enclosed by $\Sigma_u$ and $\Sigma$, $N_{\Sigma_u}$ and $N_\Sigma$ be the outward unit normal vector field of $\Sigma_u$ and $\Sigma$ respectively.
    Then by divergence theorem, we have
    \begin{align*}
        0&=\int_\Omega\div_{\mathbb{R}^n}X\d\Vol_\Omega\\
        &=\int_{\Sigma_u}\langle X,N_{\Sigma_u}\rangle\d{A_{\Sigma_u}}+\int_\Sigma\langle X,N_\Sigma\rangle\d{A_\Sigma}.
    \end{align*}
    Since $N_{\Sigma_u}=\frac{(\D u,-1)}{\sqrt{1+|\D u|^2}}$, we have $\langle X,N_{\Sigma_u}\rangle=-1$, hence by Cauchy--Schwarz inequality,
    \begin{align*}
        A[\Sigma_u]=\int_\Sigma\langle X,N_\Sigma\rangle\d{A_\Sigma}\leq\int_\Sigma|X||N_\Sigma|\d{A_\Sigma}=A[\Sigma],
    \end{align*}
    where $|X|=|N_\Sigma|=1$.
\end{proof}

Finally we have a characterization of immersed minimal surfaces in Euclidean space.
\begin{prop}
    Let $f:\Sigma^k\to\mathbb{R}^n$ ($k<n$) be an immersion, denote $\Delta f=(\Delta f^1,\cdots,\Delta f^n)$, then we have
    \[\Delta f=\mathbf{H}.\]
\end{prop}
\begin{proof}
    We use $\D$ to denote the Levi--Civita connection of $\mathbb{R}^n$ (i.e.\ directional derivative) and $\nabla$ for $\Sigma$.
    First, we notice that this proposition is local, thus we can prove it on a sufficiently small coordinate neighborhood $W$ on which $f$ is an embedding.
    Moreover, notice that we have an identification
    \[f_*Y=Yf=Yf^i\frac{\partial}{\partial{x^i}}=(Yf^1,\cdots,Yf^n)\in T\mathbb{R}^n,\]
    thus we have $Y=Yf\in\Gamma(Tf(\Sigma))$.
    Now let $\{X_i\}$ be an orthonormal frame on $W$, we have
    \begin{align*}
        \Delta f&=\sum_iX_i(X_if)-(\nabla_{X_i}X_i)f\\
        &=\sum_iX_i(X_i)-\nabla_{X_i}X_i\\
        &=\sum_i\D_{X_i}X_i-\nabla_{X_i}X_i\\
        &=\sum_i\II(X_i,X_i)\\
        &=\mathbf{H}.\qedhere
    \end{align*}
\end{proof}

\begin{cor}
    An submanifold in Euclidean space is a minimal surface if and only if its every coordinate function is harmonic.
\end{cor}

\begin{cor}
    There is no compact minimal submanifold without boundary in Euclidean space.
\end{cor}
\begin{proof}
    Since the submanifold is minimal, all its coordinate functions are harmonic.
    But the submanifold is compact, the coordinate functions reach their maximum and minimum.
    Hence by Liouville's theorem, every coordinate functions is constant.
    This is impossible.
\end{proof}