\chapter{Curvature and Transformation Groups}

In this chapter, we will discuss the space forms which we defined by Lie groups in detail.
We will show that they are constant curvature, and we will prove their uniqueness up to isometry.
After this, we will state and prove Myers--Steenrod Theorem and its rigidity result.

\section{Space Forms}

Our goal of this section is to calculate the curvature of space forms.
It's hard to calculate space forms directly using Lie group definition: this relies on the curvature formulae of Riemannian submersions.
To overcome this difficulty, we recall the models of space forms, and use these models to calculate their curvature.

\subsection*{Models of Space Forms}

Recall the space forms we introduced in Example~\ref{model space forms}:
\begin{itemize}
    \item $\mathbb{R}^n$,
    \item $\mathbb{S}^n$: $(x^1)^2+\cdots+(x^{n+1})^2=1$ in $\mathbb{R}^{n+1}$,
    \item $\mathbb{H}^n$: upper component of $(x^1)^2+\cdots+(x^n)^2-(x^{n+1})^2=-1$ in $\mathbb{R}^{n,1}$.
\end{itemize}
We will adopt these three models and equip them with Riemannian metrics.

For $\mathbb{R}^n$, we have canonical (flat) metric $\delta=(\d{x^1})^2+\cdots+(\d{x^n})^2$.

For $\mathbb{S}^n$, let $\iota:\mathbb{S}^n\to\mathbb{R}^{n+1}$ be the natural embedding, we have the canonical metric $g=\iota^*\delta$, where $\delta$ is the canonical metric of $\mathbb{R}^{n+1}$.

For $\mathbb{H}^n$, let $\iota:\mathbb{H}^n\to\mathbb{R}^{n,1}$ be the natural embedding.
Denote $\ell=(\d{x^1})^2+\cdots+(\d{x^n})^2-(\d{x^{n+1}})^2$ be the Lorentz metric of $\mathbb{R}^{n,1}$, we have the canonical metric of $\mathbb{H}^n$ to be $g=\iota^*\ell$.
We need to show that $g$ is a Riemannian metric.
\begin{lem}
    Let $\ell$ be the Lorentz metric of $\mathbb{R}^{n,1}$, $\iota:\mathbb{H}^n\to\mathbb{R}^{n,1}$ be the natural embedding.
    Then $\iota^*\ell$ is a Riemannian metric.
\end{lem}
\begin{proof}
    Clearly we only need to show $\iota^*\ell$ is positive definite.
    Let $(x^1,\cdots,x^{n+1})\in\mathbb{H}^n$, we consider its tangent vector $v=(v^1,\cdots,v^{n+1})$.
    This tangent vector satisfies equation
    \[v^1x^1+\cdots+v^nx^n-v^{n+1}x^{n+1}=0.\]
    We calculate the norm (in fact, we cannot call $\langle v,v\rangle$ norm now) of $v$, by Cauchy's inequality, we have
    \begin{align*}
        \langle v,v\rangle&=(v^1)^2+\cdots+(v^n)^2-(v^{n+1})^2\\
        &=(v^1)^2+\cdots+(v^n)^2-\frac{1}{(x^{n+1})^2}(v^1x^1+\cdots+v^nx^n)^2\\
        &\geq((v^1)^2+\cdots+(v^n)^2)\left(1-\frac{(x^1)^2+\cdots+(x^n)^2}{(x^{n+1})^2}\right)\\
        &=\frac{1}{(x^{n+1})^2}((v^1)^2+\cdots+(v^n)^2)\\
        &\geq 0.
    \end{align*}
    Clearly, when equality holds, we have $v=0$.
    Thus $\iota^*\ell$ is a Riemannian metric.
\end{proof}

\subsection*{Curvature of Space Forms}

Clearly, as $\mathbb{R}^n$ is equipped with flat metric, it has constant curvature $0$.
We write this into a proposition.
\begin{prop}
    For all $p\in\mathbb{R}^n$, we have $\Sect_p{\Pi}=0$ for any plane $\Pi\subset T_p\mathbb{R}^n$.
\end{prop}

To calculate the curvature of $\mathbb{S}^n$ and $\mathbb{H}^n$, we need to introduce the concept of totally geodesic submanifold.

\begin{defn}
    Let $M$ be a Riemannian manifold.
    A submanifold $\Sigma\subset M$ is called a \emph{totally geodesic submanifold} if the second fundamental form vanishes identically.
    Clearly this is equivalent to every geodesic of $M$ with initial vector in $T_p\Sigma$ is contained in $\Sigma$.
\end{defn}

By the very definition and Gauss equation, we have
\begin{lem}
    If $\Sigma$ is a totally geodesic submanifold of $M$, then for $p\in\Sigma$ and plane $\Pi\subset T_p\Sigma$, the sectional curvatures with respective to $\Sigma$ and $M$ are equal.
\end{lem}

A nontrivial result is
\begin{prop}\label{Fix S totally geodesic}
    Let $(M,g)$ be a complete Riemannian manifold.
    Let $S\subset\Isom(M,g)$ be a subset, define $\operatorname{Fix}{S}$ be the fixed points of $S$.
    Then any component of $\operatorname{Fix}{S}$ is totally geodesic submanifold of $M$.
\end{prop}
\begin{proof}
    Let $p\in\operatorname{Fix}{S}$, $V\subset T_pM$ be the Zariski tangent space, that is,
    \[V:=\{v\in T_pM:\ f_*|_p(v)=v,\ \forall f\in\operatorname{Fix}{S}\}.\]
    $V$ is well-defined since all $f$ in $\operatorname{Fix}{S}$ fixes $p$.
    Let $v\in V$, then the initial data of geodesic $\gamma_v(t)=\exp_p(tv)$ are fixed by $S$.
    Since isometry preseves geodesics, we have $\gamma_v\subset\operatorname{Fix}{S}$.
    Thus the proposition is proved as soon as we have shown that $\operatorname{Fix}{S}$ is a submanifold.

    Above argument shows $\exp_p$ is defined on the whole $V$.
    Let $\varepsilon$ so small that $\exp_p:B(0,\varepsilon)\to B(p,\varepsilon)$ is a diffeomorphism.
    Then by Proposition~\ref{geodesic locally length-min}, every geodesic starts from $p$ in $B(p,\varepsilon)$ is length-minimizing.
    Let $q\in\operatorname{Fix}{S}\cap B(p,\varepsilon)$, then there exists a unique geodesic $\gamma$ starts from $p$ joining $p$ and $q$.
    Since $S$ fixes the initial and end points of $\gamma$, and $S$ does not change $\gamma$'s length, by the length-minimizing property of $\gamma$, we have $\gamma$ fixed by $S$.
    Thus $\gamma\subset\operatorname{Fix}{S}$.
    This means $\exp_p$ is bijective on $V\cap B(0,\varepsilon)$, hence $\operatorname{Fix}{S}$ is a submanifold.
    The proposition is thus proved. 
\end{proof}
\begin{rem}
    The first paragraph of above proof does not use the completeness assumption, thus if $\operatorname{Fix}{S}$ is a curve, above argument can show that $\operatorname{Fix}{S}$ is a geodesic.
    This can occur when $M$ fails to be complete.
    Consider $\mathbb{R}^2\backslash\{0\}$ equipped with pullback metric of $(\mathbb{R}^2,\delta)$ and $f$ be the reflection with respective to the punctured $x$-axis.
    Then $f$ is an isometry and fixes the punctured $x$-axis, which is a union of two geodesics.
\end{rem}

Also we need Theorema Egregium in the following form.

\begin{thm}[Theorema Egregium]
    Let a $M$ be a $2$-dimensional Riemannian manifold (i.e.\ a Riemannian surface), $p\in M$.
    If the metric tensor has expression $\d{s^2}=E\d{u^2}+G\d{v^2}$ near $p$, then the sectional curvature at $p$ is
    \[\Sect_p(T_pM)=-\frac{1}{\sqrt{EG}}\left(\frac{\partial}{\partial{u}}\left(\frac{\partial_u\sqrt{G}}{\sqrt{E}}\right)+\frac{\partial}{\partial{v}}\left(\frac{\partial_v\sqrt{E}}{\sqrt{G}}\right)\right)\]
\end{thm}

We will not compute this equation here;
For an elementary proof, see~\cite[Equation~2.131]{Toponogov}.

Now we calculate the curvature of $\mathbb{S}^n$ and $\mathbb{H}^n$.

For $\mathbb{S}^n$, we see that it's a closed subspace of complete metric space $\mathbb{R}^{n+1}$, hence is complete.
Consider the orthogonal transformation 
\[A=\operatorname{diag}(1,1,1,-1,\cdots,-1),\]
we have that $A$ fixes $(x^1)^2+(x^2)^2+(x^3)^2=1$.
Hence by Proposition~\ref{Fix S totally geodesic}, we only need to calculate the sectional curvature of $\mathbb{S}^2$ (this already shows $\mathbb{S}^n$ has constant curvature, since a Riemannian surface can only have one sectional curvature).
Choose a coordinate
\[(u,v)\mapsto(\cos{u}\cos{v},\cos{u}\sin{v},\sin{u}),\]
it's known the metric tensor under this coordinate is
\[\d{s^2}=\d{u^2}+\cos^2{u}\d{v^2}.\]
Using Theorema Egregium, we have
\begin{align*}
    \Sect_p(T_p\mathbb{S}^2)&=-\frac{1}{\sqrt{\cos^2{u}}}\left(\frac{\partial{}}{\partial{u}}\left(\frac{\partial_u\sqrt{\cos^2{u}}}{\sqrt{1}}\right)+\frac{\partial{}}{\partial{v}}\left(\frac{\partial_v\sqrt{1}}{\sqrt{\cos^2{u}}}\right)\right)\\
    &=1.
\end{align*}
Thus we obtain
\begin{prop}
    $\mathbb{S}^n$ has constant sectional curvature $1$.
\end{prop}

For $\mathbb{H}^n$, first consider the hyperbola
\[(x^1)^2-(x^{n+1})^2=-1,\]
we claim it is a geodesic.
Let $\gamma(t)=(\cosh{t},0,\cdots,0,\sinh{t})$ ($t\geq 0$) be the parametrization of the hyperbola, denote whose initial vector by 
\[v=(1,0,\cdots,0).\]
Since $\mathbb{R}^{n,1}$ has flat metric, we have
\begin{align*}
    \D_{\dot\gamma(t)}\dot\gamma(t)&=\frac{\d{\dot\gamma}(t)}{\d{t}}\\
    &=(\cosh{t},0,\cdots,0,\sinh{t}).
\end{align*}
The tangent space $T_{\gamma(t)}\mathbb{H}^n$ is defined by the equation
\[x^1\cosh{t}=x^{n+1}\sinh{t},\]
whose orthochronous complement is spanned by $(\cosh{t},0,\cdots,0,\sinh{t})$, which is $\D_{\dot\gamma(t)}\dot\gamma(t)$.
Hence $\nabla_{\dot\gamma(t)}\dot\gamma(t)=0$, $\gamma$ is a geodesic.
Let $w\in T_0\mathbb{H}^n$, we can rotate $w$ to $v$, and the rotation can be extended to a orthochronous transformation of $\mathbb{R}^{n,1}$ since it's restricted in $\mathbb{R}^n\times\{0\}$.
This orthochronous transformation carries $\gamma$ to a geodesic with initial vector $w$, and can be defined on the whole $\mathbb{R}_{\geq 0}$.
Thus by Hopf--Rinow Theorem, $\mathbb{H}^n$ is complete.

After proving $\mathbb{H}^n$ is complete, we can calculate its curvature using totally geodesic submanifold.
The orthochronous transformation
\[A=\operatorname{diag}(1,1,-1,\cdots,-1,1)\]
fixes $(x^1)^2+(x^2)^2-(x^{n+1})^2=-1$.
Consider the coordinate
\[(u,v)\mapsto(\cosh{u}\cos{v},\cosh{u}\sin{v},\sinh{u}),\]
one can calculate that the expression of metric tensor under this coordinate is
\[\d{s^2}=\d{u^2}+\cosh^2{u}\d{v^2}.\]
Using Theorema Egregium, we have
\begin{align*}
    \Sect_p(T_p\mathbb{H}^2)&=-\frac{1}{\sqrt{\cosh^2{u}}}\left(\frac{\partial{}}{\partial{u}}\left(\frac{\partial_u\sqrt{\cosh^2{u}}}{\sqrt{1}}\right)+\frac{\partial{}}{\partial{v}}\left(\frac{\partial_v\sqrt{1}}{\sqrt{\cosh^2{u}}}\right)\right)\\
    &=-1.
\end{align*}
Thus we obtain
\begin{prop}
    $\mathbb{H}^n$ has sectional curvature $-1$.
\end{prop}

\section{Uniqueness of Space Forms}

In this section we prove that constant sectional curvature spaces are unique in some sense.
We have some comments for this.

First, let $(M,g)$ be a Riemannian manifold with constant sectional curvature $K$, we can scale $g$ by a positive constant $C$ to make $(M,Cg)$ into a space with constant sectional curvature $0,1,-1$, depending on the sign of $K$.
Therefore we only need to consider spaces with sectional curvature $0,1,-1$.

Second, we cannot have uniqueness unless we have some topological restriction.
Let's consider any Riemannian covering with total space to be a space form, then the base space has constant curvature.
This enlights us that we may consider universal covering of constant curvature spaces.

Third, the uniqueness is up to an isometry.
For instance, $\mathrm{O}(n+1)/\mathrm{O}(n)$ as a homogeneous space may be considered different from $\mathbb{S}^n$ ontologically, but they are isometric.

Taking all above into consideration, we have the following theorem.
\begin{thm}
    Let $M^n$ be a Riemannian manifold with constant sectional curvature of $0,1,-1$, then the universal covering of $M$ must be isometric to space forms $\mathbb{R}^n$, $\mathbb{S}^n$ or $\mathbb{H}^n$.
\end{thm}