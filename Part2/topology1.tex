\chapter{Curvature and Topology I}
In this chapter, we will disucss some topological results derived by curvature restrictions.
We will follow the order of Ricci curvature with positive lower bound, nonpositive sectional curvature and constant sectional curvature.
The corresponding results are Bonnet--Myers Theorem, Cartan--Hadamard Theorem and classification theorem of space forms.
More results on topology by curvature restrictions involves comparison theorem, which will be discussed later.

From this chapter, we will assume that all Riemannian manifolds are complete.

\section{Bonnet--Myers Theorem}

Bonnet--Myers Theorem asserts that a Riemannian manifold with positive lower bound for Ricci curvature is bounded, hence compact.
Precisely, we have
\begin{thm}[Bonnet--Myers]
    Let $(M^n,g)$ be a complete Riemannian manifold with $\Ric\geq(n-1)Kg>0$ (here we lowered the index of Ricci tensor), then $\operatorname{diam}(M)\geq\pi/\sqrt{K}$.
    In particular, $M$ is compact.
\end{thm}
\begin{proof}
    We may scale the metric to make $K=1$.
    Use contradiction argument, we may assume there exists $p,q\in M$ with $d(p,q)>\pi$.
    By completeness, there exists a unit speed geodesic $\gamma:[0,a]\to M$ such that $\gamma(0)=p$, $\gamma(a)=q$, so $d(p,q)=a>\pi$, $\gamma$ is length-minimizing.
    Let $\{E_i(t)\}$ be a parallel orthonormal frame along $\gamma$ with $E_n(t)=\dot\gamma(t)$, define
    \[U_i(t)=\sin\left(\frac{\pi t}{a}\right).\]
    Let $I$ be the index form on $\gamma$, then we have
    \begin{align*}
        \sum_{i=1}^{n-1}I(U_i,U_i)&=\sum_{i=1}^n\int_0^a\langle\dot U_i,\dot U_i\rangle-\langle R(\dot\gamma,U_i)\dot\gamma,U_i\rangle\d{t}\\
        &=\sum_{i=1}^{n-1}\int_0^a-\langle\ddot{U}_i,U_i\rangle-\langle R(U_i,\dot\gamma)U_i,\dot\gamma\rangle\d{t}\\
        &=\sum_{i=1}^{n-1}\int_0^a\frac{\pi^2}{a^2}\sin^2\left(\frac{\pi t}{a}\right)-\sin^2\left(\frac{\pi t}{a}\right)\langle R(E_i,\dot\gamma)E_i,\dot\gamma\rangle\d{t}\\
        &=\int_0^a\left((n-1)\frac{\pi^2}{a^2}-\langle\Ric(\gamma),\gamma\rangle\right)\sin^2\left(\frac{\pi t}{a}\right)\\
        &\leq\int_0^a(n-1)\left(\frac{\pi^2}{a^2}-1\right)\sin^2\left(\frac{\pi t}{a}\right)\d{t}\\
        &<0,
    \end{align*}
    the second equality is integration by parts.
    But $\gamma$ is a length-minimizing geo\-desic, $I$ must be positive definite, hence
    \[\sum_{i=1}^{n-1}I(U_i,U_i)\geq 0,\]
    this is a contradiction.
\end{proof}

\begin{rem}
    \begin{enumerate}[(1)]
        \item If we know a priori Cheng's maximal diameter theorem, the above proof can be regarded as a comparison between Jacobi fields on $M$ and sphere.
        Thus this is a prototype of our proof to Rauch's comparison theorem.
        \item The condition cannot be weakened to $K=0$, in fact, $\Sect>0$ is not enough.
        The surface $z=x^2+y^2$ in $\mathbb{R}^3$ is an counterexample.
    \end{enumerate}
\end{rem}

\begin{cor}
    Under above settings, the universal covering $\pi:\tilde{M}\to M$ is compact.
    Moreover, $\pi_1(M)$ is finite.
\end{cor}
\begin{proof}
    We can lift the metric of $M$ to make $\pi$ a Riemannian covering, then $\pi^*g$ has also a Ricci lower bound, and we can apply Bonnet--Myers Theorem.
    For the next claim, let $p\in M$, then $\pi^{-1}(p)$ is a discrete closed set in $\tilde{M}$, hence must be finite since $\tilde{M}$ is compact.
    Then the covering map is of finite sheet, $\pi_1(\tilde{M})$ has finite index in $\pi_1(M)$.
    But $\pi_1(\tilde{M})$ is trivial, hence $\pi_1(M)$ is finite.
\end{proof}

\section{Cartan--Hadamard Theorem}

Cartan--Hadamard Theorem gives the topological characterization of a Riemannian manifold with nonpositive sectional curvature.

\begin{thm}[Cartan--Hadamard]
    Let $M^n$ be a complete Riemannian manifold with $\Sect\leq 0$, then for any $p\in M$ the exponential map $\exp_p:T_pM\to M$ is a universal covering.
\end{thm}