\chapter{Curvature and Topology I}
In this chapter, we will disucss some topological results derived by curvature restrictions.
We will follow the order of Ricci curvature with positive lower bound, nonpositive sectional curvature and constant sectional curvature.
The corresponding results are Bonnet--Myers Theorem, Cartan--Hadamard Theorem and classification theorem of space forms.
More results on topology by curvature restrictions involves comparison theorem, which will be discussed later.

From this chapter, we will assume that all Riemannian manifolds are complete.

\section{Bonnet--Myers Theorem}

Bonnet--Myers Theorem asserts that a Riemannian manifold with positive lower bound for Ricci curvature is bounded, hence compact.
Precisely, we have
\begin{thm}[Bonnet--Myers]
    Let $(M^n,g)$ be a complete Riemannian manifold with $\Ric\geq(n-1)Kg>0$ (here we lowered the index of Ricci tensor), then $\operatorname{diam}(M)\geq\pi/\sqrt{K}$.
    In particular, $M$ is compact.
\end{thm}
\begin{proof}
    We may scale the metric to make $K=1$.
    Use contradiction argument, we may assume there exists $p,q\in M$ with $d(p,q)>\pi$.
    By completeness, there exists a unit speed geodesic $\gamma:[0,a]\to M$ such that $\gamma(0)=p$, $\gamma(a)=q$, so $d(p,q)=a>\pi$, $\gamma$ is length-minimizing.
    Let $\{E_i(t)\}$ be a parallel orthonormal frame along $\gamma$ with $E_n(t)=\dot\gamma(t)$, define
    \[U_i(t)=\sin\left(\frac{\pi t}{a}\right).\]
    Let $I$ be the index form on $\gamma$, then we have
    \begin{align*}
        \sum_{i=1}^{n-1}I(U_i,U_i)&=\sum_{i=1}^n\int_0^a\langle\dot U_i,\dot U_i\rangle-\langle R(\dot\gamma,U_i)\dot\gamma,U_i\rangle\d{t}\\
        &=\sum_{i=1}^{n-1}\int_0^a-\langle\ddot{U}_i,U_i\rangle-\langle R(U_i,\dot\gamma)U_i,\dot\gamma\rangle\d{t}\\
        &=\sum_{i=1}^{n-1}\int_0^a\frac{\pi^2}{a^2}\sin^2\left(\frac{\pi t}{a}\right)-\sin^2\left(\frac{\pi t}{a}\right)\langle R(E_i,\dot\gamma)E_i,\dot\gamma\rangle\d{t}\\
        &=\int_0^a\left((n-1)\frac{\pi^2}{a^2}-\langle\Ric(\gamma),\gamma\rangle\right)\sin^2\left(\frac{\pi t}{a}\right)\\
        &\leq\int_0^a(n-1)\left(\frac{\pi^2}{a^2}-1\right)\sin^2\left(\frac{\pi t}{a}\right)\d{t}\\
        &<0,
    \end{align*}
    the second equality is integration by parts.
    But $\gamma$ is a length-minimizing geo\-desic, $I$ must be positive definite, hence
    \[\sum_{i=1}^{n-1}I(U_i,U_i)\geq 0,\]
    this is a contradiction.
\end{proof}

\begin{rem}
    \begin{enumerate}[(1)]
        \item If we know a priori Cheng's maximal diameter theorem, the above proof can be regarded as a comparison between Jacobi fields on $M$ and sphere.
        Thus this is a prototype of our proof to Rauch's comparison theorem.
        \item The condition cannot be weakened to $K=0$, in fact, $\Sect>0$ is not enough.
        The surface $z=x^2+y^2$ in $\mathbb{R}^3$ is an counterexample.
    \end{enumerate}
\end{rem}

\begin{cor}
    Under above settings, the universal covering $\pi:\tilde{M}\to M$ is compact.
    Moreover, $\pi_1(M)$ is finite.
\end{cor}
\begin{proof}
    We can lift the metric of $M$ to make $\pi$ a Riemannian covering, then $\pi^*g$ has also a Ricci lower bound, and we can apply Bonnet--Myers Theorem.
    For the next claim, let $p\in M$, then $\pi^{-1}(p)$ is a discrete closed set in $\tilde{M}$, hence must be finite since $\tilde{M}$ is compact.
    Then the covering map is of finite sheet, $\pi_1(\tilde{M})$ has finite index in $\pi_1(M)$.
    But $\pi_1(\tilde{M})$ is trivial, hence $\pi_1(M)$ is finite.
\end{proof}

\section{Cartan--Hadamard Theorem}

Cartan--Hadamard Theorem gives the topological characterization of a Riemannian manifold with nonpositive sectional curvature.

\begin{thm}[Cartan--Hadamard]
    Let $M^n$ be a complete Riemannian manifold with $\Sect\leq 0$, then for any $p\in M$ the exponential map $\exp_p:T_pM\to M$ is a universal covering.
\end{thm}

We decompose the proof into several lemmas.

\begin{lem}[Ambrose]\label{Ambrose}
    Let $\varphi:(M,g)\to(N,h)$ be a local isometry.
    If $(M,g)$ is complete, then $\varphi$ is a Riemannian covering map.
\end{lem}
\begin{proof}
    We first show that $\varphi$ is surjective.
    Since $\varphi$ is a local isometry, $\varphi(M)$ is open in $N$.
    Let $y\in N$ be a limit point of $\varphi(M)$, then there exists a $x\in M$ such that there is a geodesic $\sigma:[0,1]\to N$ connecting $\varphi(x)$ and $y$.
    Since $M$ is complete, this geodesic can be lifted to $\tilde\sigma:[0,1]\to M$, and clearly $\varphi(\tilde\gamma(1))=y$.
    Hence $\varphi$ is surjective.

    Now let $x\in M,x'\in N$ with $\varphi(x)=x'$.
    Choose $\delta$ so small that $\exp_{x'}:B'(\delta)\to B_{x'}(\delta)$ is a diffeomorphism, where $B'(\delta)=\{v\in T_{x'}N:\ |v|<\delta\}$.
    Since $\varphi$ is a local isometry, $\varphi^{-1}(x')$ is discrete, say $\{x_i\}_{i\in I}$.
    For each $i\in I$, define
    \begin{gather*}
        B^i(\delta):=\{v\in T_{x_i}M:\ |v|<\delta\},\\
        B^i_\delta:=\{y\in M:\ d(y,x_i)<\delta\}.
    \end{gather*}
    We now show the following three claims sequentially:
    \begin{enumerate}[(1)]
        \item $\varphi^{-1}(B_{x'}(\delta))=\bigcup_{i\in I}B^i_\delta$;
        \item For each $i\in I$, $\varphi:B^i_\delta\to B_{x'}(\delta)$ is a diffeomorphism;
        \item If $i\neq j$, then $B^i_\delta$ and $B^j_\delta$ are disjoint.
    \end{enumerate}

    For (1), $\bigcup_{i\in I}B^i_\delta\subset\varphi^{-1}(B_{x'}(\delta))$ is clear.
    Suppose $z\in\varphi^{-1}(B_{x'}(\delta))$, then there exists a unique geodesic $\gamma:[0,1]\to N$ contained in $B_{x'}(\delta)$ such that $\gamma(0)=x'$, $\gamma(1)=\varphi(z)$ (combine Proposition~\ref{geodesic locally length-min}~and~Corollary~\ref{join points}).
    Since $M$ is complete, we can lift $\gamma$ to $\tilde\gamma$.
    Then by $\varphi(\tilde\gamma(0))=x'$, we have some $\tilde\gamma(0)=x_i$ for some $i\in I$.
    Moreover, since $L[\tilde\gamma]=L[\gamma]<\delta$, we have $z\in B^i_\delta$.
    Hence $\varphi^{-1}(B_{x'}(\delta))\subset\bigcup_{i\in I}B^i_\delta$.
    This is (1).

    For (2), since $M$ is complete, $\exp_{x_i}:B^i(\delta)\to B^i_\delta$ is well-defined.
    By naturality of exponential map (Proposition~\ref{exp natural}), we have the following commutative diagram:
    \[\begin{tikzcd}
        B^i(\delta) \ar[r, "{\d\varphi}"] \ar[d, "{\exp_{x_i}}"] & B'(\delta) \ar[d, "{\exp_{x'}}"] \\
        B^i_\delta \ar[r, "\varphi"] & B_{x'}(\delta).
    \end{tikzcd}\]
    Then
    \[\varphi\circ\exp_{x_i}=\exp_{x'}\circ\d\varphi\]
    is a diffeomorphism.
    Since $\varphi$ is a local isometry, $\exp_{x_i}$ is an immersion, but $\exp_{x_i}$ is surjective, hence $\exp_{x_i}$ is a diffeomorphism.
    Therefore $\varphi=\exp_{x'}\circ\d\varphi\circ(\exp_{x_i})^{-1}$ is a diffeomorphism.
    This is (2).

    For (3), let $z\in B^i_\delta\cap B^j_\delta$ for some $i,j$, then let $\gamma_i:[0,1]\to M$ and $\gamma_j:[0,1]\to M$ be the unique geodesics contained in $B^i_\delta$ and $B^j_\delta$ with $\gamma_i(0)=\gamma_j(0)=z$, $\gamma_i(1)=x_i$ and $\gamma_j(1)=x_j$.
    Let $\zeta:[0,1]\to N$ be the unique geodesic contained in $B_{x'}(\delta)$ with $\zeta(0)=\varphi(z)$, $\zeta(1)=x'$.
    Then by uniqueness, we have $\varphi\circ\gamma_i=\varphi\circ\gamma_j$.
    However, since $\varphi$ is a local isometry, we have
    \[\dot\gamma_i(0)=\d\varphi^{-1}(\dot\zeta(0))=\dot\gamma_j(0)\]
    and
    \[L[\gamma_i]=L[\zeta]=L[\gamma_j],\]
    hence $\gamma_i(1)=\gamma_j(1)$, that is, $x_i=x_j$.
    This is (3).
\end{proof}

\begin{lem}\label{C-H 1}
    Let $(M,g)$ be a Riemannian manifold, $p\in M$.
    If $M$ contains no conjugate points of $p$, then $\exp_p:T_pM\to M$ is a covering map.
\end{lem}
\begin{proof}
    Since $p$ has no conjugate points, $\exp_p$ is nonsingular on $T_pM$, hence is an immersion.
    Then we can equip $T_pM$ with pullback metric $\exp_p^*g$.
    If we can show $(T_pM,\exp_p^*g)$ is complete, then by Lemma~\ref{Ambrose}, $\exp_p$ is a covering map.
    Let $\gamma:[0,a)\to T_pM$ be a length-minimizing geodesic, then since $\exp_p$ is a local isometry between $(T_pM,\exp_p^*g)$ and $(M,g)$, $\exp_p\circ\gamma:[0,a)\to M$ must be a geodesic.
    By the definition of exponential map, we have $\exp_p\circ\gamma(t)=\exp_p(\dot\gamma(0)t)$, then it clearly can be extended to $a$.
    Thus by Theorem~\ref{Hopf-Rinow}~(4), $(T_pM,\exp_p^*g)$ is complete.
    This proves the lemma.
\end{proof}

\begin{lem}\label{C-H 2}
    Let $(M,g)$ be a complete Riemannian manifold with $\Sect\leq 0$, then for any $p\in M$, $p$ contains no conjugate points.
\end{lem}
\begin{proof}
    Only need to show for any geodesic $\gamma:[0,+\infty)\to M$ with $\gamma(0)=p$, the nontrivial Jacobi field $J$ along $\gamma$ with $J(0)=0$ has no zeros besides $0$.
    Consider the function $f(t)=\langle J(t),J(t)\rangle$, then we have
    \[\dot{f}(t)=2\langle J(t),\dot{J}(t)\rangle,\]
    and
    \begin{align*}
        \ddot{f}(t)&=2(\langle\dot{J}(t),\dot{J}(t)\rangle+\langle J(t),\ddot{J}(t)\rangle)\\
        &=2(\langle\dot{J}(t),\dot{J}(t)\rangle-\langle R(\dot\gamma,J)\dot\gamma,J\rangle(t)\\
        &\geq\langle\dot{J}(t),\dot{J}(t)\rangle.
    \end{align*}
    Hence
    \begin{align*}
        \dot{f}(t)&\geq\int_{0}^{t}\langle\dot{J}(t),\dot{J}(t)\rangle\d{t}+\dot{f}(0)\\
        &=\int_{0}^{t}\langle\dot{J}(t),\dot{J}(t)\rangle\d{t}+\langle\dot{J}(0),J(0)\rangle\\
        &=\int_{0}^{t}\langle\dot{J}(t),\dot{J}(t)\rangle\d{t}\\
        &>0,
    \end{align*}
    unless $\dot{J}(t)\equiv 0$ for all $t\geq 0$, but this contradicts $J$ being nontrivial.
    Hence by integrating $\dot{f}(t)$ we obtain $|J(t)|^2>0$ for $t>0$, i.e.\ $J$ has no zero other than $0$.
\end{proof}

Now Cartan--Hadamard Theorem is the corollary of Lemma~\ref{C-H 1}~and~\ref{C-H 2}.
