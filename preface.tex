\chapter{Preface}

% Key points:
% Claim Einstein summation convention.
% This note is not an encylopedia or even textbook, many proofs will not be provided.
% Smooth manifolds part is kept minimal.

I had planned to write the preface at the end of my writing.
But now I think I need a temporary preface.

We will adopt Einstein's summation convention throughout this note.
That is, if an index occurs both in the superscript and subscript, it means to take summation over this index.
For instance, we have
\[a^ib_i=\sum_{i=1}^na^ib_i.\]
If an index occurs twice in superscript or subscript, then this means something is wrong, except in the case of orthonomal frames and similar context.
For latter case, when indices actually occur twice at same position, we will not omit the summation symbol. 

This note is not an encyclopedia, or even a textbook.
Hence many proofs will not be provided.
However, I will do my best to reference these proofs.

The part on smooth manifold is kept minimal.
I only want to write about what is needed in the Riemannian geometry part.

I give special thanks to my supervisor Prof.\ Jian Ge, whose lecture notes have become a template to me.
I also give special thanks to Yifan Jin, Pengju Shao and Mingze Li, who  have continuously discussed with me and provided invaluable support.

\begin{flushright}
    \emph{Mengchen Zeng} at Wuhan, Hubei, P.\ R.\ China.\\
    October 2, 2024.
\end{flushright}