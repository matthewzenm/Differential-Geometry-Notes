\chapter{Preface}

% Key points:
% Claim Einstein summation convention.
% This note is not an encylopedia or even textbook, many proofs will not be provided.
% Smooth manifolds part is kept minimal.

I had planned to write the preface at the end of my writing, but now I think I need a temporary preface.

Roughly speaking, this note is a combination of notes on differentiable (or equivalently, smooth) manifolds and Riemannian geometry, as the GitHub description indicates.
The smooth manifold part comes from an unfinished note of mine, written in Chinese during the summer of 2022, when I began studying differential geometry.
The Riemannian geometry part comes from the lecture notes I took during the Riemannian geometry course at the 2024 BICMR summer school on differential geometry, taught by Prof.\ Chao Xia from Xiamen University.
Unfortunately, my note on differential geometry is unfinished, and my note on Riemannian geometry is full of typos and false proofs, so I had the idea to combine--or rather, rewrite--both notes.

We will adopt Einstein's summation convention throughout this note.
That is, if an index occurs both in the superscript and subscript, it means to take summation over this index.
For instance, we have
\[a^ib_i=\sum_{i=1}^na^ib_i.\]
If an index occurs twice in superscript or subscript, then this means something is wrong, except in the case of orthonormal frames and similar context.
For latter case, when indices actually occur twice at same position, we will not omit the summation symbol. 

This note is not an encyclopedia, or even a textbook.
Hence many proofs will not be provided, such as the proofs to the partition of unity theorem, the rank theorem, Stokes Theorem and so on.
However, I will do my best to reference these proofs.

The part on smooth manifolds is kept minimal.
I only want to write about what is needed in the Riemannian geometry part, so I omitted many topics such as distributions and Frobenius theorem, de Rham theory, and so on.
Also, I have not covered topology in detail, as I assume the readers are familiar with basic point-set topology and covering space theory.

\section*{Acknowledgement}

I want to give special thanks to my supervisor Prof.\ Jian Ge, whose lecture notes have become a template to me;
the special thanks also to Prof.\ Chao Xia, who led me into the realm of Riemannian geometry.
I also want to give special thanks to my shoolmates Yifan Jin, Pengju Shao and Mingze Li, who have continuously discussed with me and provided invaluable support.

\begin{flushright}
    \emph{Mengchen Zeng} at Wuhan, Hubei, P.\ R.\ China.\\
    October 2, 2024.
\end{flushright}